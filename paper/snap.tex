\documentclass[a4paper]{amsart}
%\usepackage{graphics}
%\usepackage{epsfig}
\usepackage{array}
% \usepackage{doublespace}
%\usepackage{supertabular}
\usepackage{rotating}

\setlength{\unitlength}{0.7pt}

%%%%%%%%%%%%%%%%%%%% macros %%%%%%%%%%%%%%%%%%%%

\def\A{{\mathbb A}}
\def\H{{\mathbb H}}
\def\N{{\mathbb N}}
\def\Z{{\mathbb Z}}
\def\Q{{\mathbb Q}}
\def\R{{\mathbb R}}
\def\C{{\mathbb C}}

\def\contains{\supseteq}
\def\supp{\mbox{\rm supp}}
\def\diam{\mbox{\rm diam}}
\def\path{\mbox{\rm Path}}
\def\half{{1\over 2}}

\def\cal{\mathcal}
\def\O{{\cal O}}
\def\P{{\cal P}}
\def\Otil{\tilde{\O}}
\def\Xp{{X'}}
\def\Yp{{Y'}}
\def\Zp{{Z'}}
\def\xp{{x'}}
\def\yp{{y'}}
\def\zp{{z'}}
\def\K{\tilde{K}}

\def\psl2c{\mbox{\rm PSL}(2,\C)}
\def\sl2c{\mbox{\rm SL}(2,\C)}
\def\tr{\mbox{\rm tr}}
\def\Gammatil{\tilde{\Gamma}}
\def\p{{\mathfrak p}}
\def\q{{\mathfrak q}}

\newcommand{\PreserveBackslash}[1]{\let\temp =\\#1\let\\=\temp}

%%%%%%%%%%%%%%%%% enviroments %%%%%%%%%%%%%%%%%%%

\newenvironment{pf}{\begin{proof}}{\end{proof}}

\newtheorem{theorem}{Theorem}[section]
\newtheorem{lemma}[theorem]{Lemma}
\newtheorem{proposition}[theorem]{Proposition}
\newtheorem{corollary}[theorem]{Corollary}
\newtheorem{conjecture}[theorem]{Conjecture}
\theoremstyle{definition}
\newtheorem{definition}[theorem]{Definition}
\newtheorem{example}[theorem]{Example}
\newtheorem{procedure}[theorem]{Procedure}
\newtheorem{para}[theorem]{}
%\theoremstyle{remark}
\newtheorem{remark}[theorem]{Remark}
\newtheorem{notation}{Notation}		\renewcommand{\thenotation}{}

%\theoremstyle{definition}
%\theoremstyle{remark}

%%%%%%%%%%%%%%%%%%%%%%%%%%%%%%%%%%%%%%

\title{Computing arithmetic invariants of 3-manifolds}

\author{D. Coulson}

\author{O.A. Goodman}

\author{C.D. Hodgson}

\author{W.D. Neumann}

\address{Department of Mathematics and Statistics,
University of Melbourne, Parkville, Victoria 3052, Australia.}

\thanks{Work supported by grants from the Australian
Research Council. This paper has been submitted
to Experimental Mathematics.}

\date{\today}

\begin{document}

\maketitle

\section{Introduction}

This paper describes ``Snap'', a computer program for computing
arithmetic invariants of hyperbolic 3-manifolds. Snap is based on Jeff
Weeks's program ``SnapPea'' \cite{snappea} and the number theory
package ``Pari'' \cite{pari}. SnapPea computes the hyperbolic
structure on a finite volume hyperbolic 3-manifold numerically (from
its topology) and uses it to compute much geometric information about
the manifold. Snap's approach is to compute the hyperbolic structure
to very high precision, and use this to find an exact description of
the structure. Then the correctness of the hyperbolic structure can be
verified, and the arithmetic invariants of Neumann and Reid \cite{nr1}
can be computed. Snap also computes high precision numerical
invariants such as volume, Chern-Simons invariant, eta invariant, and
the Borel regulator.  As sources of examples both ``Snap'' and
``SnapPea'' include the Hildebrand-Weeks census of all $4,815$
orientable cusped manifolds triangulated by up to seven ideal
simplices (see \cite{HiW}), and the Hodgson-Weeks census of $11,031$ low-volume closed
orientable manifolds having no geodesic of length less than $0.3$ (see \cite{Ho-We}).
(``SnapPea'' also includes a census of nonorientable cusped manifolds.)
Snap is available from http://www.ms.unimelb.edu.au/\~{}snap.

\section{Ideal Triangulations}

\def\Ct{\tilde{\C}}

SnapPea and Snap represent an orientable, finite volume, hyperbolic
3-manifold as a set of ideal tetrahedra in $\H^3$ with face
pairings. Identifying the sphere at infinity of $\H^3$ with $\Ct = \C
\cup \{\infty\}$, the orientation preserving congruence class of a 
tetrahedron is given by the cross ratio of its vertices; oriented
tetrahedra, whose vertices are numbered consistently with the
orientation, correspond to cross ratios with positive imaginary part.
After choosing orderings for the vertices of each tetrahedron, 
the tetrahedra are given by complex numbers
$\{z_1,\ldots,z_n\}$, called their {\em shape parameters}, lying in
the upper half plane.
Changing the vertex ordering of a tetrahedron may replace
$z_j$ by $1-z_j^{-1}$ or $(1-z_j)^{-1}$.


For the result of gluing these tetrahedra to represent a hyperbolic
3-manifold, the following {\em gluing conditions} must be satisfied:
\begin{enumerate}
\item Around each edge of the complex, the sum of the dihedral angles must
be $2\pi$, and the edge must be glued to itself without translation. 
\item Each cusp (neighborhood of an ideal vertex) must
either (i)  have a horospherical torus cross section, or (ii)
admit a compactification by adding a closed geodesic around which there
is an angle of $2\pi$ and no translation. 
\end{enumerate}

\begin{remark}
A probably more familiar situation is that of gluing the faces of a
compact polytope to obtain a closed geometric manifold. In this case
the translation condition is unnecessary since it is automatically
satisfied. 
\end{remark}

If every cusp has a horospherical torus cross section, the glued
complex is a complete hyperbolic 3-manifold. 
If some cusps require compactification, the result is a Dehn filling
of the glued complex. 
Ideal triangulations are described in much more detail in
\cite{thurston}. 

% SnapPea solves logarithmic gluing equations. 

The above conditions are equivalent to a set of equations in the $z_i$
which we shall describe shortly. 
First we need to define a kind of ``complex dihedral
angle'' for the edges of an ideal tetrahedron.
For each edge of an ideal tetrahedron, there is a loxodromic
transformation, having the edge as axis, and 
taking one of the two adjacent faces onto the other. The {\em
logarithmic edge parameter} of the edge is $r + i\theta$, where $r$ is
the translation distance of the transformation, and $\theta$ is the
angle through which it rotates. For oriented tetrahedra, with
consistently numbered vertices, we can take $\theta \in (0,\pi)$. The
corresponding {\em edge parameter} is $e^{r+i\theta}$.  If the
tetrahedron has shape parameter $z$, each edge parameter is one of
$z, 1-z^{-1}$, or $(1-z)^{-1}$.

Condition~1 is that, for each edge of the 3--complex, the sum of the
logarithmic edge parameters is $2\pi i$.  Condition~2 can be similarly
expressed; the exact set of terms which are added depends on whether
the cusp is 
complete or filled.
We call these the {\em
logarithmic gluing equations} of the ideal triangulation.  

When SnapPea is given a 3-manifold topologically, as a set of face
pairings for ideal tetrahedra, and perhaps also Dehn fillings for some
of the cusps, it attempts to solve the logarithmic gluing equations numerically.
A solution is called {\em geometric} if all the $z_i$ lie in the upper
half plane. Corresponding ideal tetrahedra
can then be glued together, along some of the faces, to give an ideal
fundamental polyhedron for the manifold; the remaining face pairings
give a faithful representation of its fundamental group into $\psl2c$. 

If not all of the $z_i$ lie in the upper half plane the solution may
still have a meaningful interpretation. Regard {\em any} quadruple of
points in $\Ct$ as a tetrahedron. Call it {\em geometric} if the cross
ratio lies in the upper half plane, {\em flat} if it is real and not
equal to $0$ or $1$, {\em degenerate} if it is $0, 1$, $\infty$ or
undefined (i.e.\ if two or more vertices coincide), and {\em negatively
oriented} if it is in the lower half plane.  A solution without
degenerate tetrahedra certainly gives a representation of the
fundamental group of the manifold into $\psl2c$. However, the representation
need not however be faithful and may not have a discrete image.

It follows from the existence of canonical ideal cell decompositions
of finite volume hyperbolic manifolds \cite{ep} that every such 3-manifold
can be represented using only geometric and flat tetrahedra: decompose
each cell into tetrahedra, then match differing face triangulations
using flat tetrahedra (if necessary). It is conjectured that in fact 
only geometric tetrahedra are needed in this case. 

For closed hyperbolic 3-manifolds the situation is less
clear. Certainly every such manifold can be obtained topologically by
Dehn filling a suitable hyperbolic link complement.  This means that
any solution of the gluing equations will give a representation of
the fundamental group into $\psl2c$. Unless, however, the solution
is geometric, it cannot be guaranteed that the representation is
faithful or discrete. 

% Shapes are algebraic. 

What is important, for present purposes, is that the gluing conditions can
also be given as a set of polynomial equations, with rational coefficients,
in the $z_i$. The {\em gluing equations} are obtained from the logarithmic gluing
equations by exponentiation. These equations state
 that certain products of edge parameters (of
the form $z_i, 1-z_i^{-1}$, and $(1-z_i)^{-1}$) equal $1$. Multiplying
through by suitable powers of $z_i$ and $(1-z_i)$ we obtain \
polynomial equations. Note
that the gluing equations only specify that the angle sum, around each
edge or filled cusp, is a multiple of $2\pi$. In terms of numerical
computation however, it is straightforward to check if a solution
actually gives an angle sum of precisely $2\pi$.

Mostow-Prasad rigidity \cite{mostow} implies that the solution set of
the gluing equations is 0-dimensional. It follows that the $z_i$ in
any solution are algebraic numbers: compare Macbeath's proof of Theorem~4.1
in \cite{macbeath}. For example, the complement in $S^3$ of the figure 8
knot has an ideal triangulation by two tetrahedra with shape
parameter
$$
z_1 = z_2 = \frac{1}{2} + \frac{\sqrt 3}{2}i.
$$
This is actually the shape parameter of a {\em regular} ideal tetrahedron. 

We can also assume that the entries of a set of $\psl2c$ matrices
for the fundamental group are algebraic: position the
fundamental polyhedron such that one tetrahedron has three of its vertices at
$0,1$, and $\infty$. The remaining vertices will be algebraic, as
will entries of the face pairing transformation matrices. The other
matrices, being products of these, will also have algebraic entries. 


\section{Computation with Algebraic Numbers}

In order to give an exact representation of a 3-manifold we clearly
need a way to represent algebraic numbers. We give a brief discussion,
referring to \cite{cohen} and \cite{pohst} for more details.
The most obvious way to
represent an algebraic number is to give a polynomial with rational
coefficients, whose roots include the number in
question, and somehow specify which root is intended. The latter can be done
by giving the root numerically to sufficient precision. The roots can
also be sorted and given by number.

Carrying out the field operations with algebraic numbers given in this
way is slightly non-trivial: a ``resultant trick'' enables us, given
two numbers, to compute a polynomial whose roots include the sum of
the two numbers. We must then determine which root is the sum, perhaps
by computing the latter numerically to sufficient
precision. Differences, products and quotients can be similarly
computed.

In fact we do not use quite this approach. We specify one number,
$\tau$ say, in
the above manner, then represent other numbers as $\Q$-polynomials 
in $\tau$. Let $f$ be the minimum polynomial of $\tau$ and let $n$ be
the degree of $f$. Then the field $\Q(\tau)$ is a degree $n$ extension 
of $\Q$, and
each element of $\Q(\tau)$ has a unique representation as a
$\Q$-polynomial in $\tau$ of degree at most $n-1$. 

Field operations in $\Q(\tau)$ are now very easy: sum and difference
computations are obvious; a product can be computed directly then
reduced to a polynomial of degree at most $n-1$ by subtracting a
suitable multiple of $f(\tau)$. A quotient  $g_1(\tau)/g_2(\tau)$ is
computed by using the Euclidean algorithm to find polynomials $a,b$ 
such that $af +
bg_2 = 1$, whence $b(\tau) = g_2(\tau)^{-1}$.

Pari \cite{pari} implements this kind of arithmetic: the expression {\tt
mod($g,f$)}, called in Pari a {\em polymod}, represents $g(\tau)$
where $\tau$ is a root of $f$. Note that it is not necessary to
specify {\em which} root is chosen to do arithmetic with polymods, since
a change of root is a field isomorphism.


Of course if we want to add $\alpha, \beta$ belonging to different
number fields we must either fall back on the first approach, or find
a new primitive element, $\sigma$ such that $\Q(\sigma) \supseteq
\Q(\alpha,\beta)$, and re-express both $\alpha$ and $\beta$ in terms of
$\sigma$. For the most part, however, our approach is to first find a
number field which contains all the numbers we are
interested in and then carry out the required computations inside this
field.

% finding exact shape parameters. 

Our aim then, given a 3-manifold with shape parameters
$\{z_1,\ldots,z_n\}$, is to find an irreducible polynomial $f \in
\Z[x]$ with root $\tau$ such that $z_1,\ldots,z_n \in \Q(\tau)$. In
outline what we do is this:
\begin{enumerate}
\item compute each $z_i$ to high precision,
typically around 50 decimal places; 
\item use an algorithm, called the
LLL algorithm, to guess a polynomial in $\Z[x]$ vanishing on each
$z_i$;
\item check if all the $z_i$ belong to the field generated by one of
them, also using the  LLL algorithm. 
\end{enumerate}
Step~1 is done by Newton's method, using the solution provided by
SnapPea as a starting point. 
Usually the check in Step~3 is successful. When it is not we try small
rational linear combinations of the $z_i$ to find a primitive element
for $\Q(z_1,\ldots,z_n)$. A side effect of Step~3 is that we obtain
an expression for each of the $z_i$ in terms of the primitive element. 

Since the LLL algorithm is fundamental we describe a little further
what it is and how it is applied in Steps~2 and 3 above. Most of what
follows is described much more precisely in \cite{cohen} and
\cite{pohst}.

The LLL algorithm is an algorithm which finds a ``good'' basis for an 
integer lattice
with respect to a given inner product. A good basis is one which
consists of short and approximately orthogonal elements. 
Roughly, how it does this is
to apply Gram-Schmidt ``orthogonalization'' to the starting basis, 
but modified so that only nearest integer multiples of basis
elements are added or subtracted. 
Whenever
an element is obtained which is significantly shorter than the
preceeding ones, it is moved in front of them, and 
Gram-Schmidt is started again from there. 
The resulting basis always contains elements not too far from
being shortest in the lattice. We emphasize that the result is
dependent on the {\em inner product}: the lattice in our case is
always simply the integer lattice $\Z^n$; it is by varying the
inner product that we obtain useful results. 

Now suppose that $z$ approximates an algebraic number $\tau$. To find
an integer polynomial, of degree at most $m$, vanishing on $\tau$ we
look for one which is small on $z$. In fact we use LLL to find a short
vector in $\Z^{m+1}$ with respect to the inner product given by the
quadratic form\footnote{A slightly different
quadratic form is actually used, namely the $a_0^2$ term is omitted if
$z$ is real and the $a_0^2$ and $a_1^2$ terms are omitted if $z$ is
nonreal. The reason is pragmatic: LLL initializes by doing a
true Gram-Schmidt reduction of the form, and the resulting
basis-change is the same to within machine precision for the modified
form, but is given by a much simpler formula.}: 
$$ 
(a_0,\ldots,a_m) \mapsto a_0^2 + \ldots + a_m^2 + N
| a_0 + a_1 z + a_2 z^2 + \ldots + a_m z^m |^2, 
$$ 
where $N$ is a large number, around $10^{1.5 d}$ if $z$ is given to
$d$ decimal places. If $a_0 + a_1 z + a_2 z^2 + \ldots + a_m z^m$ is
not zero, to approximately the precision to which $z$ is known, the
term $N | a_0 + a_1 z + a_2 z^2 + \ldots + a_m z^m |^2$ will make
$(a_0,\ldots,a_n)$ long. Thus if LLL finds any short vectors,
it has most likely found $(a_0,\ldots,a_m)$ such that $a_0 + a_1 \tau + 
a_2 \tau^2 + \ldots + a_m \tau^m = 0$. By factoring this polynomial, 
and identifying which irreducible factor has
$\tau$ as a root, we can determine $\tau$'s minimum polynomial. 
Of course, if $\tau$'s
minimum polynomial has degree greater than $m$, this whole process is doomed
to failure. Assuming however that we have chosen $m$ sufficiently large, this
application of LLL completes Step~2 above. 

For Step~3 we need to check if $\alpha$, algebraic, approximated by
$w$, belongs to $\Q(\tau)$. We use the LLL algorithm to find a small
vector in $\Z^{n+1}$ with respect to the inner product given by the 
quadratic form:
$$ 
(a,a_0,\ldots,a_{n-1}) \mapsto a^2 + a_0^2 + \ldots + a_{n-1}^2 + N
| aw + a_0 + a_1 z + \ldots + a_{n-1} z^{n-1} |^2, 
$$ 
where $N$ is as before and $n$ is the degree of $\tau$'s minimum
polynomial. As before, if LLL finds a short vector, 
it most likely has found $(a,a_0,\ldots,a_{n-1})$ such
that $a\alpha + a_0 + a_1 \tau + a_2 \tau^2 + \ldots + a_{n-1} \tau^{n-1} = 0$. 
Since $n$ is the degree of $\tau$'s minimum polynomial, $a$ should not
be zero; so we obtain an expression for $\alpha$ in terms of $\tau$. 
On the other hand, if $aw + a_0 + a_1 z + a_2 z^2 + \ldots + a_{n-1}
z^{n-1}$ is not zero, to approximately the precision to which $z$ and
$w$ are known, it is likely that $\alpha \not\in \Q(\tau)$.
Refinements of these procedures can be found in \cite{cohen} and \cite{pohst}. 

We can give a rough analysis of the above use of the LLL algorithm.
Denote by $b(a_0,\dots,a_m)$ the above quadratic form that is reduced
by the LLL algorithm to find a good integer polynomial for $z$.  One
can easily check that this bilinear form has determinant approximately
equal to $N$ or $N^2$ according as $z$ is real or non-real. (In this
discussion, ``approximately equal to'' will mean ``equal to a bounded
multiple of.'' The actual determinants are very close to $N$ and
$(\operatorname{Im}zN)^2$ respectively.)  In our applications $z$ is
complex, but we will analyze the algorithm without this assumption, so
let $k=1$ or $2$ according as $z$ is real or non-real.  Putting
$N=10^P$, we can write $$\det(b)\sim 10^{kP}.$$ Now crude estimates
suggest that a ``random'' quadratic form of determinant $D$ on
$\Z^{m+1}$ will have minimum on $\Z^{m+1}-\{0\}$ approximately equal
to $D^{1/(m+1)}$. In our case:
$$\min\{b(a_0,\dots,a_m):(a_0,\dots,a_m)\in\Z^{m+1}-\{0\}\}\sim
10^{kP/(m+1)}.$$ Since the coefficients $a_i$ contribute their squares
to this minimal $b(a_0,\dots,a_m)$, they will be bounded by
approximately $10^{kP/(2m+2)}$.  Thus if we expect coefficients
bounded by $10^c$ we need $c$ less than $kP/(2m+2)$ and hence
$$P\approx 2(m+1)c/k$$ or larger.  Conversely, once $P$ is chosen, $c$
is bounded by about $Pk/2(m+1)$.

The minimal $b(a_0,\dots,a_m)$ also includes a contribution $10^Pl^2$ with
$$l=a_0+a_1z+\dots+a_mz^m,$$
so we also have 
$10^P|l|^2\sim 10^{kP/(m+1)}$ so
$$|l|\sim 10^{P(k-m-1)/2(m+1)}.$$
This is expected even if the original $\tau$ that $z$ approximates
does not satisfy an integer polynomial in degree $m$.  Thus to detect
that the polynomial that we find is a good one, we should use somewhat
more than $P(m+1-k)/2(m+1)\approx P/2$ digits of precision. 

Snap adjusts $P$ so that it uses $d=2P/3$ digits of precision.  Since
$k=2$ in Snap's applications, this means we can hope Snap will find
polynomials with coefficients up to about $10^{3d/2(m+1)}$.  Snap's
default (which can be changed at any time) is to work with degree $16$
and precision $d=50$, so we can hope to find polynomials with
coefficients bounded by about $10^{4.5}$, and expect to find them if the
coefficients are significantly smaller than this.

We can roughly quantify the likelihood of finding ``false positives''
in these applications of LLL. Given $n$ random complex numbers
$\zeta_1,\dots,\zeta_n$ in the unit disk, the number of complex
numbers of the form $a_1\zeta_1+\dots+a_n\zeta_n$ in the unit disk
with $|a_i|\le 10^c$ is approximately $10^{(n-2)c}$, so the total area
covered by a disk of radius $10^{-p}$ around each will be
approximately $10^{(n-2)c-2p}\pi$ if $p$ is significantly larger
than $(n-2)c$.  Thus the probability of one of these linear
combinations $a_1\zeta_1+\dots a_n\zeta_n$ being ``accidentally''
within $10^{-p}$ of $0$ is about $10^{(n-2)c-2p}$. With a machine
precision of $10^{-d}$ and coefficients up to $10^c$ we should take
$p=d-c$, so the likelihood of a false positive becomes about
$10^{nc-2d}$.  With Snap's defaults ($n=17, d=50$) described above and
$c=4.5$, this is about $10^{-23}$.

As we increase both precision and degree, the running time of the
algorithm goes up. 
We were unable to find any estimates of the expected
running time of the LLL algorithm in the literature, but experiment suggests
 that typical running times using Pari 2.03's implementation
on a Sparc~5 machine satisfy:
$${\rm~running~time~(sec)~}\approx 
3.7 \times 10^{-7} (precision)^{2.6} (degree)^{2.7},$$
for degrees between $10$ and $20$ and precisions between $80$ and $180$.

Finally, note that whatever choice we make for $n$ in Step~2,
it is the degree of the minimum polynomial found which governs $n$ in
Step~3: often this will be smaller and the LLL computations in Step~3
will run correspondingly faster.

Snap follows the procedure outlined above to find a number field
containing all the shape parameters of a given 3-manifold, and an
exact expression for each shape in terms of a primitive element for that
field. Snap's ``verify'' function then substitutes the exact shapes
back into the gluing equations to check that they are satisfied. Here
is sample output for the figure 8 knot complement. 

\begin{verbatim}
Shapes (Numeric)
shape(1) = 0.50000000000000000000000 + 0.86602540378443864676372*i
shape(2) = 0.50000000000000000000000 + 0.86602540378443864676372*i

Shape Field
min poly: x^2 - x + 1
root number: 1
numeric value of root: 0.50000000000000000000000 + 0.8660254037844
3864676372*i

Shapes (Exact)
shape(1) = x 1.33737 E-67
shape(2) = x 5.24561 E-68

Gluing Equations
Meridians:
1, 0; 0, 1; 0 -> 1 : 9.27301 E-68
Longitudes:
0, -2; 0, 4; 2 -> 1 :  0.E-57
Edges:
2, -1; -1, 2; 0 -> 1 : 1.29822 E-67
-2, 1; 1, -2; 0 -> 1 : 1.29822 E-67
\end{verbatim}

The root number says which root of the minimum polynomial is used as a
primitive element for the field. The numbering scheme used will be
described when we discuss canonical representations of number fields
in $\C$. The small number following each exact shape 
(eg.\ {\tt 1.33737 E-67}) gives the accuracy of the 
originally computed numerical shape. It is included 
only as a sanity check.

Finally we have the gluing equations. As we have already
noted, the gluing equations come down to the requirement that certain
products of terms of the form $z_i, 1 - z_i^{-1}$ and $(1 - z_i)^{-1}$
give 1. This is equivalent to certain products of powers of
$z_i, 1-z_i$ and $-1$, giving 1: see for example \cite{nz}. Reading
each gluing equation horizontally we have powers of $z_1,\ldots,z_n$,
powers of $1-z_1,\ldots,1-z_n$, and the power of $-1$, followed by
their product in exact arithmetic.
This is followed after a colon by the precision to which the
logarithmic gluing equation has been verified.  Since the 
gluing equation is exactly correct, the logarithmic
gluing equation is known to be correct up to an integer multiple of $2\pi
i$, so it would suffice to verify it to much lower precision than 
is actually done.  

Since this output shows that the logarithmic gluing equations have
been verified exactly, and the shape parameters were in the upper half
plane, signifying correctly oriented simplices, it proves the
existence of a hyperbolic structure with an ideal triangulation with
the given simplex shapes.

The meridian and longitude referred to in the printout are curves, in
a cross section of the cusp, which give a basis for the first homology group
of that cross section. Typically SnapPea uses shortest curve and next
shortest independent curve, in the Euclidean structure on a horospherical cusp
cross section, as the meridian and longitude respectively.
(For knot and link complements, SnapPea uses the conventional 
terminology: where a meridian 
means a curve bounding a disk transverse to the knot or link, while a
longitude means a curve that
runs parallel to the knot or to a component of the link and is
null-homologous in $S^3$ minus the knot or link component.)
Corresponding to each meridian or longitude is a gluing equation for
the cusped hyperbolic structure. The gluing equations for a Dehn
filled manifold include  one equation for each filled
cusp, corresponding to the filling  curves. 

\section{Commensurability Invariants}

Two finite volume, orientable, hyperbolic 3-manifolds are said to be {\em
commensurable} if they have a common finite-sheeted cover. Subgroups
$\Gamma,\Gamma' \subset \psl2c$ are commensurable if there exists
$g\in \psl2c$ such that $g^{-1}\Gamma g \cap \Gamma'$ is a finite
index subgroup of both $g^{-1}\Gamma g$ and $\Gamma'$. Therefore, by
Mostow rigidity, finite
volume, orientable, hyperbolic 3-manifolds are commensurable if and only
if their fundamental groups are commensurable as subgroups of $\psl2c$. 

\subsection{The Invariant Trace Field}

Let $\Gamma$ be the group of covering transformations of such a
manifold, and let $\Gammatil$ denote the preimage of $\Gamma$ in $\sl2c$.
The traces of elements of $\Gammatil$
generate a number field $\Q(\tr\Gamma)$ called the {\em trace field} of
$\Gamma$. That $\Q(\tr\Gamma)$ is a number field follows from the
observation that $\Gamma$ is finitely generated and, by conjugating
suitably (as described at the end of section 2)
 we can assume that the generators have algebraic entries. 
The trace field $\Q(\tr\Gamma)$ is almost, but not quite, a
commensurability invariant of $\Gamma$: see \cite{reid}.

The {\em invariant trace field} $k(\Gamma)$ of $\Gamma$ may be defined as
the intersection of all the fields $\Q(\tr\Gamma')$, as $\Gamma'$ varies
over all finite index subgroups of $\Gamma$. Defined in this way it is
clear that $k(\Gamma)$ is a commensurability invariant of $\Gamma$.
What is less clear is that it is ever non-trivial. We have, however, the
following. 

\begin{theorem}[Reid \cite{reid}]
$k(\Gamma) = \Q(\{\tr^2(\gamma) \mid \gamma\in\Gamma\}) =
\Q(\tr\Gamma^{(2)})$, where $\Gamma^{(2)}$ is the finite index subgroup of
$\Gamma$ generated by squares $\{\gamma^2\mid\gamma\in\Gamma\}$. 
\end{theorem}

We have seen how it is possible, given a set of generators for a field,
to guess a primitive element for that field along with its
corresponding minimum polynomial. In order to compute the trace and
invariant trace fields of $\Gamma$ we must find {\em finite} sets of
generators for the two fields. 

\begin{theorem} \label{genset}
Let $\Gammatil\subset\sl2c$ be finitely generated by $\{g_1,\ldots,g_n\}$.
The trace, $\tr(g_{i_1}\ldots g_{i_k})$, of an element of
$\Gammatil$ can be expressed as a
polynomial with rational coefficients in the traces: $\tr(g_i),
1\leq i \leq n$, $\tr(g_ig_j), 1\leq i < j\leq n$, and (if $n>2$) the
trace of one triple product of generators, e.g.\ $\tr(g_1g_2g_3)$.
Also, $\tr(g_{i_1}\ldots g_{i_k})$ is an algebraic integer if $\tr(g_i),
1\leq i \leq n$, and $\tr(g_ig_j), 1\leq i < j\leq n$ are algebraic integers. 
\end{theorem}

\begin{proof}
For the trace relations used in the following, see Magnus
\cite{magnus}. 
Let $K$ be the field generated over $\Q$ by the traces $\tr(g_i),
1\leq i \leq n$, and $\tr(g_ig_j), 1\leq i < j\leq n$. Let $P_{ijk} =
\tr(g_i g_j g_k) + \tr(g_i g_k g_j)$ and $Q_{ijk} = \tr(g_i g_j g_k) .
\tr(g_i g_k g_j)$. 
Then $\tr(g_i g_j g_k)$ and $\tr(g_i g_k g_j)$ are the roots of $z^2 -
P_{ijk} z + Q_{ijk} = 0$. 
Fricke's Lemma (in \cite{magnus}) implies that
$P_{ijk}$ and $Q_{ijk}$ are integer
polynomials in the $\tr(g_i)$ and $\tr(g_ig_j)$, hence
they are in $K$. 
Writing $\Delta(g_i,g_j,g_k)$ for the discriminant $P_{ijk}^2 - 4 Q_{ijk}$ it
is clear that for any extension $K_1$ of $K$,
$\tr(g_i g_j g_k) \in
K_1$, if and only if both $\tr(g_i g_j g_k)$ and $\tr(g_i g_k g_j)\in
K_1$, if and only if $\sqrt{\Delta(g_i,g_j,g_k)}\in K_1$. 

By \cite{magnus}, Lemma~2.3, for any $i,j,k$ and $i',j',k'$ in
$\{ 1,\ldots,n \}$, 
$$
\sqrt{\Delta(g_i,g_j,g_k)} \cdot \sqrt{\Delta(g_{i'},g_{j'},g_{k'})} \in K.
$$
Therefore $\sqrt{\Delta(g_i,g_j,g_k)} \in K_1$ if and only if
$\sqrt{\Delta(g_{i'},g_{j'},g_{k'})}\in K_1$. 
If we now put $K_1 = K(\tr(g_1g_2g_3))$ it follows from the above
observations that $\tr(g_i g_j g_k) \in K_1$ for all $i,j,k$ in
$1,\ldots,n$. 

We show next, by induction on $k\geq 3$, that $K_1$ contains the
traces of all $k$--fold products of the generators $g_i$. We have just
shown that this is so for $k=3$. Suppose then that $k>3$ and $K_1$
contains the traces of all $(k\!-\!1)$--fold products of generators. 
Then for each product
$g' = g_{i_1} \ldots g_{i_{k-2}}$, $K_1$ contains all traces, and
all traces of products of pairs, of elements
in the set $\{g_1,\ldots,g_n,g'\}$.
Moreover it contains at least one triple product, namely $\tr(g_1 g_2
g_3)$. By the above argument it follows that $K_1$ contains the traces
of all triple products of elements of this set. In particular, $K_1$
contains the trace of $g'g_i g_j$ for each $g'$ as above, and $i,j$ in
$\{ 1,\ldots,n \}$. Since these are all the $k$--fold products of the $g_i$,
this proves the first statement. 

Finally, if the $\tr(g_i)$ and $\tr(g_ig_j)$ are all algebraic
integers, $P_{ijk}$ and $Q_{ijk}$ are also. Therefore $\tr(g_i g_j
g_k)$ and $\tr(g_i g_k g_j)$, being roots of a monic polynomial with
algebraic integer coefficients, are again integral. The same induction 
argument then
shows that all traces of $k$--fold products of the $g_i$ 
are in the ring of integers of $K_1$.
\end{proof}

Theorem \ref{genset} enables us to compute the trace field of $\Gamma
= <\! g_1,\ldots,g_n\! >$. There is no particularly obvious set of
generators for $\Gamma^{(2)}$ which we can use to compute the
invariant trace field of $\Gamma$. Fortunately, Corollary~3.2 of
\cite{hlm} tells us that $\Q(\tr\Gamma^{(2)}) = \Q(\tr\Gamma^{SQ})$ where
$\Gamma^{SQ} = <\!  g_1^2,\ldots,g_n^2\! >$, as long as $\tr(g_i)\neq
0$, for $i=1,\ldots,n$.

Snap computes trace fields and invariant trace fields in much the same
way that it computes a field containing all the shape parameters. It
first computes high precision numeric expressions for a set of generators
for the group of covering transformations of a manifold. Then it uses
LLL to find a primitive element in terms of which the appropriate set
of traces can be expressed. 

For example: $(6,1)$--Dehn filling on the figure 8 knot complement
yields a closed hyperbolic 3-manifold with volume
$1.284485300468\ldots$. Its group of covering transformations is $<\!
g_1, g_2\!>$ with
\begin{gather*}
g_1 \approx 
\left( \begin{matrix}
-1.135368+0.572291 i & 0.0 \\
0.702328+0.354014 i & -0.702328-0.354014 i
\end{matrix}\right),\\
g_2\approx
\left( \begin{matrix}
-1.226699+1.467712 i & 2.689343+1.705870 i \\
-0.265154+0.168189 i & 0.0
\end{matrix}\right).
\end{gather*}

Snap prints the invariant trace field as follows:
\begin{verbatim}
 Invariant trace field
 minumum polynomial: x^3 + 2*x - 1
 root number: 2
 numerical value of root: -0.2266988257582018 + 1.467711508710224*i
 signature: [1, 1]
 discriminant: -59
  ...
\end{verbatim}
It also gives exact expressions for the traces used to generate this
field: 
\begin{verbatim}
 Invariant trace field generators
 tr(g1^2) = mod(-x^2 - x - 1, x^3 + 2*x - 1)
 tr(g2^2) = mod(x^2 - 2*x - 1, x^3 + 2*x - 1)
 tr(g1^2g2^2) = mod(-x + 2, x^3 + 2*x - 1)

\end{verbatim}

This is all very well but there is not much point in computing
invariants, like the invariant trace field, if we cannot compare two
and decide whether they are the same or different. Simple invariants
of a number field include its degree (dimension over $\Q$), which is
equal to the degree of the minimum polynomial of any primitive
element, and its signature $(r_1,r_2)$, where  $r_1$ is the number of 
real roots, and $r_2$ is the number of conjugate
pairs of non-real roots of a minimum polynomial. 

We also have the {\em discriminant}. The algebraic integers of a
number field $\Q(\tau)$ form a free $\Z$--submodule of $\Q(\tau)$
of rank $[\Q(\tau):\Q]$.  The bilinear map $(x,y)\mapsto \tr(xy)$
gives a nondegenerate inner product on $\Q(\tau)$ as a $\Q$--vector
space. Given any basis of the ring of integers we form the matrix of
inner products of basis elements, taken pairwise. The determinant of
this is in $\Z$ and is independent of the choice of basis. It is
called the discriminant of the number field. 

In fact we can construct a {\em canonical minimum polynomial} 
which is a complete isomorphism invariant for number fields. 
The so-called $T_2$ norm of
a number field is given by the
the inner product
$$(x,y) \mapsto  \sum_{i=1}^n \sigma_i(x)\overline{\sigma_i(y)},$$
where  $\sigma_1 , \ldots, \sigma_n$ are the embeddings of
the number field $\Q(\tau)$ into $\C$, and the bar denotes ordinary 
complex conjugation. 
This gives a positive definite inner product
on $\Q(\tau)$, and we can enumerate integers of $\Q(\tau)$ in order of
their $T_2$ norm. The set of integers of smallest norm 
that generate $\Q(\tau)$
is canonical. Their minimum polynomials include one which
is lexicographically first, and this serves as a canonical
minimum polynomial.  See \cite{cohen} for further discussion.

The trace fields we compute are not just abstract number fields, they
are actually subfields of $\C$. Complex conjugate
subfields arise from complex conjugate representations in
$\psl2c$ of the same fundamental group, and just correspond to reversing the
orientation of a hyperbolic 3-manifold.  Otherwise
different subfields mean essentially different values of the invariant. 
Since several roots of the
canonical minimum polynomial might generate the same subfield of $\C$,
we sort the roots into some fixed order and take the first which gives
the required subfield. This gives us a {\em canonical root number}
for the subfield%
\footnote{
\label{rootnumbers}
In fact, we make an {\em ordered list} of the real roots followed by the
complex roots having positive imaginary part; these are arranged in 
increasing order of real part and increasing absolute value 
of imaginary part (if real parts are equal).
We then try each real root in turn (if the field is real) or 
each complex root followed by its complex conjugate (if the field is non-real).
Finally, we assign the root a number: if the root has non-negative real
part we give its position in the list, 
otherwise we give the negative of the number for its conjugate.}.

For example: in the Hodgson-Weeks census of low-volume, closed,
hyperbolic 3-manifolds, the manifolds denoted {\hbox{m010}(-1,3)} and
{\hbox{s594}(-4,3)} have isomorphic invariant trace fields, with
canonical minimum polynomial $x^4 + x^2 - x + 1$, {\em but} they have
different canonical root numbers, namely 1 and 2
respectively. Therefore their invariant trace fields differ and they
are not commensurable.

\subsection{The invariant quaternion algebra}

\def\eqn{aX^2 + bY^2 - Z^2 = 0}

Let $K$ be a field of characteristic zero. A {\em quaternion algebra over}
$K$ is a simple central algebra of dimension 4 over $K$. These are discussed
in detail in \cite{vigneras}.
Let $(a,b)$
be a pair of nonzero elements of $K$. Up to isomorphism, there is a
unique quaternion algebra $A$ containing elements $i,j$ satisfying $i^2 =
a, j^2 = b$, and $ij = -ji$, and such that $\{1,i,j,ij\}$ form a basis
for $A$ as a $K$--vector space. Such a pair $(a,b)$ is called a {\em Hilbert
symbol} for $A$. Every quaternion algebra over $K$ has a Hilbert
symbol, but the symbol is far from being unique. 

$A$ is a division algebra if and only if the equation $\eqn$ has no
non-trivial solutions for $X,Y,Z\in K$. If $A$ is not a division
algebra, it is isomorphic with $M(2,K)$, the algebra of all $2$ by $2$ matrices
over $K$ (and conversely, the latter is not a division algebra for any
$K$). Over $\R$ there are just two quaternion algebras: the ``usual''
Hamiltonian quaternion algebra, which has Hilbert symbol $(-1,-1)$ and
is a division algebra, and $M(2,\R)$. Over $\C$ there is just
$M(2,\C)$.


As before, let $M = \H^3/\Gamma$ be a finite volume, orientable
hyperbolic 3-manifold. Let $\tilde{\Gamma}^{(2)}$ be the preimage in
$\sl2c$ of the group generated by squares of elements of 
$\Gamma \subset \psl2c$. The
{\em invariant quaternion algebra $A(\Gamma)$ of $\Gamma$}, is the
$k$--subalgebra of $M(2,\C)$ generated by $\tilde{\Gamma}^{(2)}$,
where $k$ denotes the invariant trace field of $\Gamma$. 

\begin{theorem}[See \cite{hlm}]
Let $g,h$ be non-commuting elements of $\Gamma^{(2)}$ with
$\tr(g)\neq \pm 2$. 
Then $A(\Gamma)$ has Hilbert symbol
$$
(\tr(g^2) - 2, \tr([g,h]) - 2),
$$
where $[g,h]$ denotes the commutator $ghg^{-1}h^{-1}$. 
\end{theorem}

Snap computes a Hilbert symbol for the invariant quaternion algebra of
a 3-manifold by finding $g,h\in \Gamma^{(2)}$ as above, and computing
exact expressions for $\tr(g^2) - 2$ and $\tr([g,h]) - 2$. The
non-uniqueness of the Hilbert symbol means that this is not, by itself,
enough to tell us whether or not two 3-manifolds have the same
quaternion algebra. 

The remainder of this section describes how the classification of
quaternion algebras over a number field gives a complete invariant
which we can compute. 
We fix a number field $K$, and quaternion algebra $A$ over
$K$ with Hilbert symbol $(a,b)$.


\begin{theorem}[See \cite{vigneras}]
Let $K$ and $A$ be as above. The
isomorphism class of $A$ is determined by the (finite) set of real and
finite places of $K$ at which $A$ is ramified. The total number of
places, real and finite, at which $A$ ramifies, is even. 
\end{theorem}

Recall that a {\em place} of a number field $K$ is an equivalence
class of absolute values $|.|:K\rightarrow \R$. A place is called {\em real}
(resp.\ {\em complex}) if the completion of $K$ with respect to $|.|$ is
isomorphic with $\R$ (resp.\ $\C$). The real (resp.\ complex) places
of $K$ are in one-to-one correspondence with embeddings
$\sigma:K\rightarrow \R$ (resp.\ conjugate pairs of non-real embeddings
$\sigma:K\rightarrow \C$). 

A place is called {\em finite} if it arises from a valuation
$v:K^*=K-\{ 0\} \rightarrow \Z$, 
i.e.\ there is a real number $\lambda \in (0,1)$
such that $|x| = \lambda^{v(x)}$ for all $x\in K^*$.
These valuations, in turn, are in one-to-one correspondence with
prime ideals of $\Z_K$, the ring of integers of $K$: if $\p$ is a
prime ideal of $\Z_K$, then for each $x\in K^*$, let $v_{\p}(x) = r$ where $r$
is the unique integer such that $x\in \p^r - \p^{r+1}$. 

\def\Kbar{\overline{K}}

For a fixed place of $K$, let $\sigma:K\rightarrow \Kbar$ denote the
embedding of $K$ into its completion. Then $A\otimes_{\sigma}\Kbar$ is
a quaternion algebra over $\Kbar$.  $A$ is said to be {\em ramified at
$\sigma$} if $A\otimes_{\sigma}\Kbar$ is a division algebra. In
general, over a complete field with absolute value (e.g.\ $\R$), there
exists at most one quaternion division algebra.

Computing the real ramification of $A$ is straightforward:
$A\otimes_{\sigma}\R$ has Hilbert symbol $(\sigma(a),
\sigma(b))$. Therefore $A$ is ramified at $\sigma$ if and only if both
$\sigma(a)$ and $\sigma(b)$ are negative.

\def\Kp{K_{\p}}

For the remainder of this section we consider the problem of computing
the finite ramification of $A$. Slightly different notation is
convenient. 
Let $\p\subset\Z_K$ be a prime ideal and let $\Kp$ denote the
corresponding completion of $K$. We regard $K$ as a
subfield of $\Kp$, omitting any explicit mention of an embedding. 
Finally, we write $A_{\p}$ for $A\otimes\Kp$. 

\begin{proposition} \label{hilsym}
Let $K$, $A$ and $(a,b)$ be as above. 
Let $\p\not\:\mid 2$ be a prime ideal of $\Z_K$. Then
$A_{\p}$ is a division algebra if and only if none of $a, b$ and $-ab$ are
squares in $\Kp$. If $a, b$ and $-ab$ all have even $\p$-adic
valuation, at least one of them is a square. 
\end{proposition}

\begin{proof}
See Lemma~II.1.10 and the table following it in \cite{vigneras}.
(Note that Vigneras uses the notation $\{a,b\}$ for our Hilbert
symbol $(a,b)$.)
\end{proof}

\def\Otil{\overline{\Z_K}}

This proposition has two useful consequences. Firstly, that the
finite ramification of $A$ is restricted to the finite set of 
primes $\p$ dividing $2ab$.
Secondly, for primes $\p$ not dividing $2$, the question of whether $A$ is
ramified reduces to determining whether
certain $c\in K$ are squares in $K_{\p}$. Proposition~\ref{qr} settles
this question for us. The proof uses Hensel's Lemma \cite[page 42]{lang},
which is valid for any prime
$\p\subset \Z_K$, and corresponding absolute value $|x | = \lambda^{v_{\p}(x)}$. 
Here, $\Otil$  refers to the closure of $\Z_K$ in $\Kp$. 

\begin{lemma}[Hensel] \label{hensel}
Let $f(X)$ be a polynomial in $\Z_K[X]$. Let $x_0$ be an element of $\Z_K$
such that $|f(x_0)| < |f'(x_0)^2|$, where $f'$ denotes the formal
derivative of $f$. Then $f$ has a root $x$ in $\Otil$ such that 
$|x - x_0| < 1$.
\end{lemma}

\begin{proposition} \label{qr}
For each $c\in K^*$ 
and prime $\p\subset\Z_K$ there exists
$w\in K^*$ such that $cw^2 \in \Z_K$ and $v_{\p}(cw^2) \in \{0,1\}$.
Suppose now $\p\not\:\mid 2$.
Then $c$ is a square in $\Kp$ if and only if $v_{\p}(cw^2) = 0$ and $cw^2$ projects to
a square in the finite field $\Z_K/\p$. 
\end{proposition}

\begin{proof}
Let $w_1\in \Z_K$ be the denominator of $c$. Then $cw_1^2\in \Z_K$.
By the Chinese Remainder Theorem, we can find an element $u\in K^*$
such that $v_{\p}(u) = -1$ while $v_{\q}(u)\geq 0$ for all prime
ideals $\q \neq \p$.  Then $w = w_1 u^m$, where 
$m$ is the integer part of  $v_{\p}(cw_1^2)/2$, 
has the required property.

Let $c' = cw^2$. Let $f(X) = X^2 - c'$.  
If $v_{\p}(c') = 0$ and $c'$ projects to a square
in $\Z_K/\p$ we can lift a square root to obtain $x_0\in \Z_K$ such
that $f(x_0)\in \p$ while $f'(x_0) = 2x_0 \not\in \p$.
Lemma~\ref{hensel} then implies that $c'$ is a square in $\Kp$.

Conversely, if $c' = x^2$ for some $x \in \Kp$, $v_{\p}(c') = 0$ and $x$
projects to a square root of the projection of $c'$ in $\Z_K/\p$. 
\end{proof}

We turn now to the case $\p\mid 2$. Recall that 
$A_{\p}$ is a division algebra if
and only if the equation $\eqn$ has no non-trivial solution for $X,Y,Z
\in \Kp$. Denote by $e > 0$ the $p$-adic valuation of $2$, i.e. $e =
v_{\p}(2)$, or equivalently, $|2| = \lambda^e$. (This $e$ is in fact the
ramification index of the field extension $K/\Q$ at $\p$.) We omit the
easy proof of the following lemma.

\begin{lemma} \label{approx}
Let $X,X'\in \Kp$ and suppose $|X| \leq 1$ and $|X - X'| \leq \lambda^k$ for
some non-negative integer $k$. Then $|X^2 - {X'}^2| \leq
\lambda^{\min\{k+e,2k\}}$.
\end{lemma}

Multiplying $a$ and $b$ by suitable squares if necessary, by the first part of
Proposition~\ref{qr}, we can assume $a,b\in \Z_K$ and $v_{\p}(a),
v_{\p}(b) \in \{0,1\}$.

\begin{proposition} \label{lifting}
Let $a,b \in \Z_K$ be such that $v_{\p}(a), v_{\p}(b) \in \{0,1\}$. 
Let $R$ be a finite set of representatives for the ring $\Z_K /
\p^{e+3}$, where $e = v_{\p}(2)$. The equation
\begin{equation}
aX^2 + bY^2 - Z^2 = 0 \label{hilbert}
\end{equation}
has a solution for $X,Y,Z \in \Kp$, if and only if there exist elements
$\Xp,\Yp,\Zp \in R$ such that $|a\Xp^2 + b\Yp^2 - \Zp^2| \leq \lambda^{2e+3}$ and
$\max\{|\Xp|,|\Yp|,|\Zp|\} = 1$.
\end{proposition}

\begin{proof}
Let $(X,Y,Z)$ be a solution of (\ref{hilbert}) in $\Kp$. Multiplying
through, if necessary, by a suitable power of a uniformizing element
$\pi$ with $v_{\p}(\pi) = 1$, 
we can assume
$X,Y,Z \in \Otil$ and $\max\{|X|,|Y|,|Z|\} = 1$. Since $\Z_K$ is dense
in $\Otil$, and $R$ is $\lambda^{e+3}$-dense in $\Z_K$, we can choose $\Xp,\Yp,\Zp
\in R$ such that $|X - \Xp|,|Y - \Yp|,|Z - \Zp| \leq \lambda^{e+3}$. By
Lemma~\ref{approx}, $|a\Xp^2 + b\Yp^2 - \Zp^2| \leq \lambda^{2e+3}$. 

Conversely, let $\Xp,\Yp,\Zp \in R$ be such that $|a\Xp^2 + b\Yp^2 - \Zp^2|
\leq \lambda^{2e+3}$ and $\max\{|\Xp|,|\Yp|,|\Zp|\} = 1$. If $|\Xp| = 1$ then
$|2a\Xp| \geq \lambda^{e+1}$, and therefore $|a\Xp^2 + b\Yp^2 - \Zp^2| <
|2a\Xp|^2$. Regarding $a\Xp^2 + b\Yp^2 - \Zp^2$ as a polynomial in $\Xp$
alone, by Lemma~\ref{hensel} there exists $X\in\Otil$ such that $aX^2 +
b\Yp^2 - \Zp^2 = 0$. The same argument applies if $|\Yp| = 1$ or $|\Zp| =
1$. Since at least one of the three cases must hold, the result follows. 
\end{proof}

Propositions~\ref{hilsym}, \ref{qr} and \ref{lifting} reduce the
task of computing the finite ramification of a quaternion algebra over
a number field to a finite number of steps. We remark that the details
of these computations are readily handled by Pari. In particular,
Pari has functions for factoring algebraic numbers and ideals into
primes, and for computing valuations. The uniformizing element and the
element $u$, invoked in the proof of Proposition~\ref{qr}, are
constituent parts of Pari's way of representing a prime ideal (and
are thus readily available).

\begin{remark}
For an invariant quaternion algebra $A = A(\Gamma)$,
the calculation of finite ramification
can sometimes be simplified by using the following observation from
\cite{gmmr}. Assume that all traces of elements in $\Gammatil$ 
are algebraic {\em integers}, and  
let $g,h$ be non-commuting loxodromic elements of $\Gammatil^{(2)}$.
Then any prime $\p$ which ramifies the quaternion algebra $A$ must 
divide $\tr([g,h]) - 2$, where $[g,h]$ denotes the commutator $ghg^{-1}h^{-1}$. 
\end{remark}

\subsection{Arithmeticity}
Finally we describe the ``arithmetic'' construction of Kleinian groups
of finite co-volume. Let $A$ be a quaternion algebra over a number
field $K$. The integers of $A$, i.e.\ elements of $A$ which have
a monic
minimum polynomial with integral coefficients over $K$, do not in
general form a subring of $A$. The analogous role in $A$, to that of
$\Z_K$ in $K$, is now played by an order of $A$. An {\em order} $\O$
of $A$ is a rank 4 $\Z_K$-submodule of the set of 
integers of $A$, containing $1_A$, and
closed as a subring of $A$. Orders always
exist but are not generally unique. The units $\O^1$ of $\O$ form a
multiplicative subgroup. For each real or complex place
$\sigma$ of $K$, $\sigma$ induces a map of $A$ into $\H$, $M(2,\R)$ or
$M(2,\C)$. If $K$ has precisely one complex place, and every real
place is ramified (i.e.\ maps $A$ into $\H$), then the image of $\O^1$
in $M(2,\C)$ is a discrete subgroup of $SL(2,\C)$ of finite
co-volume. This group is said to be {\em derived
from a quaternion algebra.} A subgroup $\Gamma$ of $SL(2,\C)$ is {\em
arithmetic} if it is commensurable with one derived from a quaternion
algebra. The $K$ and $A$ of the construction can be recovered as the
invariant trace field, and invariant quaternion algebra respectively,
of $\Gamma$. 

\begin{remark}
This is not really {\em the} definition of arithmeticity; there
is a much more general definition in the context of lattices in
semi-simple Lie groups. It is a result of Borel that the above
construction yields all the arithmetic subgroups of $SL(2,\C)$. 
\end{remark}

A result of Reid \cite{reidphd} (see also \cite{tak}, \cite{hlm}), 
shows that a discrete subgroup $\Gamma$ of $SL(2,\C)$ is
arithmetic if and only if the following conditions are satisfied:
\begin{enumerate}
\item The invariant trace field $k = \Q(\tr\Gamma^{(2)})$, has exactly
one complex place.
\item $A(\Gamma)$ is ramified at every real place of $k$.
\item $\Gamma$ has integer traces (which is equivalent to
$\tr\Gamma^{(2)} \subseteq \Z_k$).
\end{enumerate}
This enables us to determine 
whether or not hyperbolic 3-manifolds are arithmetic.
(See Tables~\ref{lowvolclosed} and \ref{algebras}
for some examples.)

Arithmetic subgroups of $\sl2c$ are commensurable if and only if
they have the same invariant quaternion algebra. Therefore the
arithmetic manifolds grouped together in Table~\ref{algebras}
are commensurable. Non-arithmetic manifolds with the same invariant
trace field, quaternion algebra and integrality or otherwise of
traces, may still be incommensurable. It is work in progress to find
a {\em computable,} complete commensurability invariant for the
non-arithmetic case. 

\begin{example}\label{twinsex}

The paper \cite{BPZ} describes an interesting family of 
hyperbolic ``twins'' --- pairs of non-homeomorphic 
closed hyperbolic 3-manifolds with the same volume. 
These examples are obtained by Dehn filling on the
manifold denoted $\hbox{m009}$ in SnapPea's notation;
this is the once-punctured torus bundle over $S^1$
with monodromy given by the matrix
$
\left[ \begin{matrix}
3 & 2 \\                                     
1 & 1                     
\end{matrix}\right]
$.
We use the geometric choice of
basis for homology of the boundary torus 
consisting of shortest geodesic and next shortest independent geodesic
on a horospherical torus cross section. 
Then the Dehn fillings $\hbox{m009}(p,q)$ and $\hbox{m009}(-p,q)$
give non-homeomorphic closed manifolds of equal hyperbolic volume,
for each pair of relatively prime integers $(p,q)$,
except for the 8 non-hyperbolic Dehn fillings $(\pm 3,1), (\pm 2,1),
(\pm 1,1), (0,1)$, and $(1,0)$.

In Problem 3.60(H) of \cite{kirby}, Pzrzytycki asked if
these pairs are commensurable. Using Snap, we find that
these pairs of manifolds generally have the isomorphic invariant trace fields,
but have different invariant quaternion algebras
so are not commensurable. However, there is one pair,
$\hbox{m009}(5,1)$ and $\hbox{m009}(-5,1)$, which are arithmetic manifolds
of volume $1.8319311883544380\ldots$
with the same invariant quaternion algebra, hence are commensurable.
Table \ref{twins} shows some arithmetic data
for the lowest volume twins. (The descriptions of invariant trace field
and quaternion algebras are explained in section \ref{tables} below.)
\end{example}

\begin{table}[h] 
\centering
\begin{tabular}{|>{\PreserveBackslash\raggedright}p{3.2cm}|l| %
	>{\PreserveBackslash\raggedright}p{3.5cm}|
	>{\PreserveBackslash\raggedright}p{3.0cm}|l|}

\hline

\hbox{Manifold} Volume  & Homology & Invariant trace field  & Quaternion algebra & Int/Ar\ \\

\hline 

%1.4140610441653916              Z/6  0.0766020666485759  0.794134662992  m009( 4, 1)    
 m009( 4, 1) 
 1.4140610441653916  &$\Z/6$  &
\hbox{$x^3 - x^2 + 1$} 
[1, 1] (2) &
\hbox{$(5, x-2)$} [1]&
$1/1$ \\

\hline 

%1.4140610441653916             Z/10 -0.1182687333152426  0.364894686526  m009(-4, 1)    
 m009(-4, 1) 
1.4140610441653916    &$\Z/10$ &
\hbox{$ x^3 - x^2 + 1$} 
[1, 1] (-2) &
\hbox{$(19, x-3)$} [1]&
$1/1$ \\

\hline 

%1.8319311883544380        Z/2 + Z/4  0.0625000000000000  0.530637530953  m009( 5, 1)    
 m009( 5, 1) 
 1.8319311883544380  &$\Z/2 + \Z/$4&
\hbox{$ x^2 + 1$} 
[0, 1] (1) &
\hbox{$(2, x+1) (5,x+2)$} [~] &
$1/1$ \\

\hline

%1.8319311883544380        Z/2 + Z/6 -0.1041666666666667  0.481211825060  m009(-5, 1)    
m009(-5, 1)
1.8319311883544380    &    $\Z/2 + \Z/6$ &
\hbox{$ x^2 + 1$} 
[0, 1] (1) &
\hbox{$(2, x+1) (5,x+2)$} [~] &
$1/1$ \\

\hline

%1.8435859723266779              Z/6  0.2233269246580219  0.477961349958  m009(-1, 2) 
m009(-1, 2)
1.8435859723266779     &         $\Z/6$ &
\hbox{$x^5 - 2x^4 - 2x^3 + 4x^2 - x + 1$} 
[3, 1] (-4) &
\hbox{$(2,x^2+x+1) (5,x+1)$} [1,2]&
$1/0$ \\


\hline
%1.8435859723266779              Z/2  0.2350064086753114  0.480433913874  m009( 1, 2)
 m009( 1, 2)
 1.8435859723266779      &        $\Z/2$  & 
\hbox{$x^5 - 2x^4 - 2x^3 + 4x^2 - x + 1$} 
[3, 1] (4) &
[1,2] &
$1/0$ \\


\hline
%1.9415030840274678             Z/10  0.2124785362605743  0.425721998618  m009(-3, 2)
m009(-3, 2)
1.9415030840274678       &      $\Z/10$  &
\hbox{$x^5 - x^4 - 2x^3 - x^2 + 2x + 2$} 
[3, 1] (4) &
\hbox{$(2,x) (19,x+2)$} [2,3]&
$1/0$ \\
    

\hline
%1.9415030840274678              Z/2  0.2458547970727590  0.464306096866  m009( 3, 2)
m009( 3, 2) 
1.9415030840274678       &     $\Z/2$ &
\hbox{$x^5 - x^4 - 2x^3 - x^2 + 2x + 2$} 
[3, 1] (-4) &
\hbox{$(2,x) (2,x^3+x^2+1)$} [2,3]&
$1/0$ \\

\hline
%2.0624516259038381             Z/10  0.0516817236776574  0.377933762621  m009( 6, 1) 
m009( 6, 1)
2.0624516259038381       &      $\Z/10$ &
\hbox{$x^5 - x^4 + x^3 + 2x^2 - 2x + 1$} 
[1, 2] (-2) &
\hbox{$(2,x+1)  (19,x+9)$} {$(2,x^3+x^2+1)$}\ \ \  [1]&
$1/0$ \\

\hline
%2.0624516259038381             Z/14 -0.0933483903443241  0.377933762621  m009(-6, 1) 
m009(-6, 1) 
2.0624516259038381       &    $\Z/14$ &
\hbox{$x^5 - x^4 + x^3 + 2x^2 - 2x + 1$} 
[1, 2] (2) &
\hbox{$(2,1+x)$} [1]&
$1/0$ \\

\hline
%2.1340163368014022             Z/14  0.2059468384760179  0.368422300349  m009(-5, 2)
 m009(-5, 2)
2.1340163368014022       &     $\Z/14$ &
\hbox{$x^5 - 3x^3 - 2x^2 + 2x + 1$}
[3, 1] (4) &
\hbox{$(71,x-11)$} [1,3]&
$1/0$ \\
    

\hline
%2.1340163368014022              Z/6 -0.2476135051426846  0.368422300349  m009( 5, 2)
m009( 5, 2) 
2.1340163368014022      &        $\Z/6$ &
\hbox{$x^5 - 3x^3 - 2x^2 + 2x + 1$}
[3, 1] (-4) &
\hbox{$(2) (5,x-2)$} [1,3]&
$1/0$    \\

\hline

\end{tabular}
\caption{A family of pairs of closed manifolds with equal volume}
\label{twins}
\end{table}

\def\snap{{Snap}}\def\snappea{{SnapPea}}
\def\Bloch{\mathcal B}
\def\Prebloch{\mathcal P}
\def\CS{\operatorname{CS}}
\def\cs{\operatorname{cs}}
\def\vol{\operatorname{vol}}
\section{Chern-Simons Invariant and Eta Invariant}

The eta-invariant $\eta(M)$ and the Chern-Simons invariant $\cs(M)$
are geometrically defined invariants of an hyperbolic $3$-manifold
$M$.  These invariants often take rational values, but are
conjecturally ``usually'' transcendental (a precise conjecture is in
\cite{neumann-yang1}).  \snap\ computes these invariants to high
precision. The Chern-Simons invariant is also computed (to lower
precision) by \snappea.  In the following two subsections we say
in more detail what these invariants are and how \snap\ computes them.

In the versions\footnote{There are two commonly used normalizations of
Chern-Simons invariant in the literature related by
$\cs(M)=\frac1{2\pi^2}\CS(M)$. Although the invariants are usually
defined for compact $M$ we allow cusps, see below.} we consider, the
eta-invariant $\eta(M)$ is a real invariant while the Chern-Simons
invariant $\cs(M)$ is defined modulo $\frac12$.  Moreover, the
Chern-Simons invariant is determined by the eta-invariant: $\cs(M)$ is
simply $\frac32\eta(M)$ (mod $\frac12$).

Why do we bother with $\cs(M)$, given that it is immediately determined by
$\eta(M)$?  A first reason is that $\cs(M)$ is somewhat easier to
compute. Secondly, $\cs(M)$ also has algebraic significance; it is
closely tied to the Bloch invariant, an algebraic/number-theoretic
invariant which we describe in the next section.

A less significant reason is that $\cs(M)$ multiplies by degree in
coverings, so it is a tool for commensurability questions. However, the
behaviour of $\eta(M)$ for coverings is also well understood (and
related to other interesting invariants, see e.g.,
\cite{atiyah-patodi-singer2, neumann1}).

\subsection{Chern-Simons Invariant}

The Chern-Simons invariant $\cs(M)$ is defined for any compact
$(4k-1)$-dimensional Riemannian manifold $M$ and is an obstruction to
conformal immersion of $M$ in Euclidean space \cite{chern-simons}. It
is the integral of a certain $(4k-1)$-form that is defined in terms of
curvature. (More generally, the Chern-Simons invariant is an invariant
of a connection on a manifold and our $\cs(M)$ is the Chern-Simons
invariant for the Riemannian connection on the tangent bundle of $M$).

For hyperbolic 3-manifolds Meyerhoff \cite{meyerhoff} extended the
definition of $\cs(M)$ to allow $M$ to have cusps. The point is that
if $M'$ is a compact manifold obtained by Dehn filling $M$ then
$\cs(M')$ is naturally the sum of a term that varies analytically on
hyperbolic Dehn filling space and a discontinuous summand
($-\frac1{2\pi}$ times the sum of torsions of the geodesics added by Dehn
filling), see \cite{nz} and \cite{yoshida}. So one defines
$\cs(M)$ as the value of the analytic term at the complete hyperbolic
structure on $M$.

This leads to an invariant $\cs(M)$ of a hyperbolic 3-manifold $M$ in
$\R/\frac12\Z$. If $M$ is closed the Chern-Simons invariant is well
defined modulo $1$, but \snap\ and \snappea\ still only compute modulo
$\frac12$.  This is no real loss, since the Chern-Simons invariant of
a closed manifold $M$ modulo $1$ can also be computed from the first
homology of $M$ together with the eta-invariant $\eta(M)$, both of
which \snap\ can also compute.

Another significance of $\cs(M)$ for a hyperbolic 3-manifold is that it
has natural analytic relation to $\vol(M)$. In fact $\vol(M) +
2\pi^2i\cs(M)$ is a natural complexification of $\vol(M)$ and the formulae
one uses to compute $\cs(M)$ give $\vol(M)$ as well.

The method of computation used by \snap\ and \snappea\ is as follows.
Recall that these programs compute using ideal triangulations. Let $M$
be a cusped hyperbolic $3$-manifold with ideal triangulation and
$M(p,q)$ the result of hyperbolic Dehn surgery on some chosen cusp of
$M$, triangulated by deformed versions of the original tetrahedra.  In
\cite{neumann2} Neumann gave a formula for $\cs(M(p,q))+\alpha$,
where $\alpha$ is a constant, in terms of the simplex parameters of
these deformed ideal tetrahedra.  The constant $\alpha$ is unknown,
but is independent of $p$ and $q$. Thus if the exact Chern-Simons
invariant is known for just one of the manifolds $M(p,q)$ then
$\alpha$ can be deduced, so $\cs(M(p,q))$ can be computed for all the
$M(p,q)$. As one computes $\cs(M)$ for more manifolds one has more
reference points to compute new families of values.  Using this
``bootstrapping'' procedure Weeks and Hodgson computed $\cs(M)$ for the
data-bases of manifolds in \snappea.  The computed values are included
in \snappea\ so that they are available for further Dehn surgeries.

In fact the constant $\alpha$ is always an integer multiple of
$1/24$ in the version of the formula that \snap\ uses (this was
conjectured in \cite{neumann2} but has since been proved, see
\cite{neumann-in-progress} or the announcement in
\cite{neumann-hilbert}). Thus \snap\ can compute the high precision
value of $\cs(M)$ up to a multiple of $1/24$ and this multiple can
then be determined from \snappea's lower precision value.  An improved
formula that computes $\cs(M)$ exactly is now known (loc.\ cit.). This
avoids the need of the bootstrapping procedure and will eventually be
implemented in \snap.

\subsection{Eta-Invariant}

The eta-invariant $\eta(M)$ is also defined for any closed oriented
Riemannian $(4k-1)$-manifold. It was initially defined by Atiyah,
Patodi and Singer as a measure of the ``asymmetry'' of the spectrum of
the Laplacian on $M$, but they proved \cite{atiyah-patodi-singer2} that
it can also be given by the following formula:
\def\sign{\operatorname{sign}}
$$\eta(M):=
\int_XL - \sign(X), 
$$
where:

\noindent$\bullet$~~~~$X$ is any Riemannian $4k$-manifold with $\partial X
= M$ such that the metric on some collar neighbourhood of
$\partial X$ is isometric to the product metric on
$M\times[0,\epsilon)$,
and 

\noindent$\bullet$~~~~$L$ is the Hirzebruch $L$-class as a
$4k$-form on $X$, defined in terms of curvature as in, for example,
the appendix to \cite{milnor-stasheff}. 

The Hirzebruch index theorem tells one that the above formula gives
zero for a closed manifold $X$ and it is then a standard argument to
see that it gives an invariant of $M$ that does not depend on the
choice of $X$ when $X$ has boundary $M$ as above.  If $k>1$ then $M$
may not be the boundary of any $X$, but the disjoint union $2M$ of $2$
copies of $M$ is a boundary, so this formula can be used to define
$\eta(2M)$, and hence define $\eta(M)$ as $\frac12\eta(2M)$.

The relation of $\eta(M)$ to $\cs(M)$ for a compact $3$-manifold $M$
is (\cite{atiyah-patodi-singer2}): $$3\eta(M) \equiv
2\cs(M)+\tau\quad\text{(mod $2$)},$$ where $\tau$ is the number of
$2$-primary summands of $H_1(M;\Z)$. Thus $\eta(M)$ completely
determine $\cs(M)$ if $M$ has known homology.  There is also a cusped
version of this --- Meyerhoff and Ouyang \cite{meyerhoff-ouyang}
extended the definition of $\eta(M)$ to cusped $M$ for which one has
chosen a basis of homology at each cusp.

A formula for $\eta(M(p,q))$ in terms of ideal triangulations for
manifolds $M(p,q)$ as described above was given in
\cite{meyerhoff-neumann}, where it was proved ``locally'' (i.e., in a
neighbourhood of the complete structure $M$ in analytic Dehn filling
space). It was proved globally in \cite{ouyang}. The formula is a
modification of Neumann's Chern-Simons formula by the addition of
certain arithmetic terms.  Again, there is an undetermined constant
that is independent of $p$ and $q$.  Thus the above bootstrapping
procedure, which will no longer be needed for computing Chern-Simons
invariant, is still needed to compute $\eta(M)$ through the tables
maintained by \snap\ and \snappea.  For a manifold $M$ which has not
yet been linked by a sequence of hyperbolic Dehn fillings and
drillings (removing closed curves) to a manifold with known
eta-invariant, \snap\ cannot compute $\eta(M)$. This still applies to
most of the knot and link complements in the standard knot and link
tables, for example.

It is conjectured that the bootstrapping procedure will always work.
That is:

\begin{conjecture}
Any two hyperbolic $3$-manifolds are related by a sequence of
hyperbolic drillings and fillings.
\end{conjecture}

\snap\ and \snappea\ provide good facilities for searching for such
sequences, so there is much experimental evidence for the conjecture.
The emphasis here is \emph{hyperbolic} drilling and filling: that is,
each drilling or filling should move between points in the appropriate
analytic Dehn filling space.  Without this restriction the conjecture
is easy, since every $3$-manifold is obtainable by Dehn surgery on
some link in the $3$-sphere.

\begin{remark}\label{CS remark}
The formula mentioned earlier for $\cs(M)$ actually computes the
Chern-Simons invariant for the natural flat connection 
on the associated principal $\psl2c$-bundle over $M$
rather than the
Riemannian connection.  It is shown by Dupont and Kamber in
\cite{dupont-kamber} that these are the same in $\R/\Z[\frac16]$.
In that paper they were considering a more general situation and not
aiming for best denominators, and Dupont informs us that their proof
works without introducing denominators in the $3$ dimensional case that
we are interested in.

The equality of the Riemannian and flat Chern-Simons invariants also
follows if one assumes the conjecture above.  Indeed, in
\cite{yoshida} the formula we use to compute Chern-Simons is proved in
the context of the Riemannian Chern-Simons invariant and in
\cite{neumann-in-progress} it is proved for the flat Chern-Simons
invariant.  Thus we have two formulae that differ at most by the
unknown constant they contain, valid over the analytic Dehn filling
space for $M$.  Thus the difference of Riemannian and flat
Chern-Simons is constant on any analytic Dehn filling space.  It is
zero for some examples, so if the bootstrapping conjecture is true,
the bootstrapping procedure shows the difference is always zero.
\end{remark}



\def\snap{{Snap}}
\def\snappea{{SnapPea}} 
\def\Bloch{\mathcal B}
\def\Borel{\operatorname{Borel}}
\def\Prebloch{\mathcal P}
\def\cs{\operatorname{cs}}
\def\rank{\operatorname{rank}}
\def\PSL{\operatorname{PSL}}

\section{Bloch Invariant and PSL-Fundamental Class} 

For details on what we discuss here see \cite{neumann-in-progress,
neumann-yang2, neumann-yang3} or the expository article
\cite{neumann-hilbert}. 

\subsection{PSL-Fundamental Class} We first describe the
``PSL-fundamental class'' of an hyperbolic $3$-manifold $M$. This is a
homology class $[M]_{PSL}$ in the homology group $H_3(\psl2c;\Z)$,
where we are taking homology of $\psl2c$ as a discrete group. If $M$
has cusps, $[M]_{PSL}$ is only well defined up to an element of order
2 in $H_3(\psl2c;\Z)$. We describe how we compute this invariant
numerically later.

Let $M=\H^3/\Gamma$ be a compact hyperbolic $3$-manifold. Then
$H_*(\Gamma;\Z)=H_*(M;\Z)$, since $M$ is a $K(\Gamma,1)$-space. Thus
$H_3(\Gamma;\Z)\simeq\Z$ with a natural generator given by the
fundamental class of $M$. The inclusion $\Gamma\to\psl2c$ induces a
map $H_3(\Gamma;\Z)\to H_3(\psl2c;\Z)$.

\begin{definition} The \emph{PSL-fundamental class}
$[M]_{PSL}\in H_3(\psl2c;\Z)$ is the image of the natural generator of
$H_3(\Gamma;\Z)$ under the above map. \end{definition}

If $M$ is non-compact the PSL-fundamental class is harder to define,
and we postpone it. It lies in $H_3(\psl2c;\Z)/C_2$, where $C_2$ is a
cyclic subgroup of $H_3(\psl2c;\Z)$ of order $2$. This cyclic subgroup
exists and is unique by the next theorem. In our notation we will
ignore this $C_2$ ambiguity and speak of $[M]_{PSL}\in H_3(\psl2c;\Z)$.

Note that we can conjugate $\Gamma$ to lie in a subgroup $\PSL(2,K)$
of $\psl2c$, where $K$ is a number field, and $[M]_{PSL}$ is then
defined in $H_3(\PSL(2,K);\Z)$ (this has only been proved modulo
torsion in the cusped case). Usually, the smallest $K$ for which one
can do this will be a quadratic extension of the trace field of
$\Gamma$ (and there are infinitely many such fields which work). The
following theorem tells us that if we work modulo torsion then we can
actually use the invariant trace field.

This theorem summarises results of various people, see
\cite{neumann-yang3} for more details.

\begin{theorem}\label{various results}
\begin{itemize}
\item[1.]
$H_3(\psl2c;\Z)$ is the direct sum of
\begin{itemize}
\item its torsion subgroup, isomorphic to $\Q/\Z$, and \item an
infinite dimensional $\Q$ vector space (conjectured to be countable).
\end{itemize}
\item[2.] If $k\subset\C$ is a number field then
$H_3(\PSL(2,k);\Z)$ is the direct sum of \begin{itemize}
\item its torsion subgroup and
\item $\Z^{r_2}$, where $r_2$ is the number of conjugate pairs of
complex embeddings of $k$.
\end{itemize} Moreover, the map $H_3(\PSL(2,k);\Z)\to H_3(\psl2c;\Z)$
is injective modulo torsion.
\item[3.] If $k$ is the invariant trace field of $M$ then some
positive multiple of $[M]_{PSL}$ is in the image of
$H_3(\PSL(2,k);\Z)\to H_3(\psl2c;\Z)$.
\end{itemize}
\end{theorem}

In fact, one can show that, after possibly adding a torsion element,
$2^{b+1}[M]_{PSL}$ is in the image of $H_3(\PSL(2,k);\Z)\to
H_3(\psl2c;\Z)$, where $b=\rank H_1(\Gamma;\Z/2)$. Moreover the
coefficient $2^{b+1}$ can be replaced by $1$ if $M$ has cusps. 

\subsection{Invariants of the PSL-fundamental class} There is a
homomorphism $$\hat c\colon H_3(\psl2c;\Z)\to\C/2\pi^2\Z$$ called the
``Cheeger-Simons class'' (\cite{cheeger-simons}) whose real and
imaginary parts give Chern-Simons invariant and volume: $$\hat
c([M]_{PSL})= 2\pi^2\cs(M)+i\vol(M)\,.$$ ($\cs(M)$ is here appearing
as the Chern-Simons invariant of the flat connection, as discussed in
Remark \ref{CS remark}). We therefore denote the homomorphisms given
in the obvious way by the real and imaginary parts of $\hat c$ by:
$$\cs\colon H_3(\psl2c;\Z)\to \R/\Z\,,\qquad
\vol\colon H_3(\psl2c;\Z)\to \R\,.$$

\begin{conjecture}\label{rama} The Cheeger-Simons class is injective.
That is, volume and Chern-Simons invariant determine elements of
$H_3(\psl2c;\Z)$ completely. This is a special case of a general
conjecture of Ramakrishnan in algebraic $K$-theory; see
\cite{neumann-hilbert} for a discussion. \end{conjecture}

If $k$ is an algebraic number field and
$\sigma_1,\dots,\sigma_{r_2}\colon k\to\C$ are its different complex
embeddings up to conjugation then denote by $\vol_j$ the composition
$$\vol_j=\vol\circ(\sigma_j)_*\colon H_3(\PSL(2,k);\Z)\to\R.$$ The map
$$\Borel:=(\vol_1,\dots,\vol_{r_2})\colon
H_3(\PSL(2,k);\Z)\to\R^{r_2}$$ is called the \emph{Borel regulator}.
\begin{theorem}\label{borel theorem} The Borel regulator maps
$H_3(\PSL(2,k);\Z)/Torsion$ injectively onto a full sublattice of
$\R^{r_2}$.
\end{theorem}

It is known that $\cs$ is injective on the torsion subgroup of
$H_3(\psl2c;\Z)$. Thus, by Theorems \ref{various results} and
\ref{borel theorem}, $\cs(M)\in\R/\Z$ and $\Borel([M]_{PSL}) \in
\R^{r_2(k)}$ determine the PSL-fundamental class $[M]_{PSL}\in
H_3(\psl2c;\Z)$ completely, where $k$ is the invariant trace field of
$M$. 

\snap\ computes $$\Borel(M):=\Borel([M]_{PSL}).$$ To describe how, it
helps to introduce the ``Bloch Group'' $\Bloch(\C)$. In the next
subsection we give this group a geometric description, but in fact, by
a result of Bloch and Wigner and others, it is naturally the quotient
of $H_3(\psl2c;\Z)$ by its torsion subgroup $\Q/\Z$. 

We can now explain how a cusped $3$-manifold has a PSL-fundamental
class in $H_3(\psl2c;\Z)$ modulo an order $2$ ambiguity. We shall see
that it has a natural class in the Bloch group, which can be thought
of as a PSL-fundamental class modulo torsion, and the Meyerhoff
definition of Chern-Simons invariant then pins down the
PSL-fundamental class up to the stated ambiguity. It would be nice to
find a more direct definition that gives a fundamental class in
$H_3(\PSL(2,K);\Z)$ (modulo a similar ambiguity to the above) when
$\Gamma\subset\PSL(2,K)$, but the above definition does not do this.


\subsection{Bloch group} There are several different definitions of
the Bloch group in the literature. They differ at most by torsion and
they agree with each other for algebraically closed fields. We shall
use the following.

\begin{definition}\label{def-bloch} Let $k$ be a field. The {\em
pre-Bloch group $\Prebloch(k)$} is the quotient of the free
$\Z$-module $\Z (k-\{0,1\})$ by all instances of the following
relations:
\begin{gather} [x]-[y]+[\frac
yx]-[\frac{1-x^{-1}}{1-y^{-1}}]+[\frac{1-x}{1-y}]=0, \label{5term}\\
[x]=[1-\frac 1x]=[\frac 1{1-x}]=-[\frac1x]=-[\frac{x-1}x]=-[1-x].
\label{invsim}
\end{gather} The first of these relations is usually called the {\em
five term relation}. The {\em Bloch group $\Bloch(k)$} is the kernel
of the map $$
\Prebloch(k)\to k^*\wedge_\Z k^*,\quad [z]\mapsto 2(z\wedge(1-z)). $$
\end{definition}

For $k=\C$, the relations (\ref{invsim}) express the fact that
$\Prebloch(\C)$ may be thought of as being generated by isometry
classes of ideal hyperbolic 3-simplices. The five term relation
(\ref{5term}) then expresses the fact that in this group we can
replace an ideal simplex on four ideal points by the cone of its
boundary to a fifth ideal point. As is shown an appendix to
\cite{neumann-yang3}, the effect is that $\Prebloch(\C)$ is a group
generated by ideal polyhedra with ideal triangular faces modulo the
relations generated by cutting and pasting along such faces. 

\subsection{The Bloch invariant}\label{Bloch Invariant} 

Suppose we have an ideal triangulation of an hyperbolic $3$-manifold
$M$ using ideal hyperbolic simplices with cross ratio parameters
$z_1,\dots,z_n$. This ideal triangulation can be a genuine ideal
triangulation of a cusped $3$-manifold, or a deformation of such a one
as used by \snap\ and \snappea\ to study Dehn filled manifolds, but it
may be of much more general type, see \cite{neumann-yang3}. 

\begin{definition} The {\em Bloch invariant $\beta(M)$} is the element
$\sum_1^n [z_j]\in\Prebloch(\C)$. If the $z_j$'s all belong to a
subfield $K\subset \C$, we may consider $\beta(M)$ as an element of
$\Prebloch(K)$. \end{definition}

It is shown in \cite{neumann-yang3} that 
\begin{theorem}\label{field restrict} If $\beta(M)$
can be defined as above in $\Prebloch(K)$ then it actually lies in
$\Bloch(K)\subset\Prebloch(K)$ and is independent of triangulation.
\end{theorem}

In these terms, the Borel regulator $\Borel(M)$ can also be thought of
as an invariant of the Bloch invariant $\beta(M)$ and can be computed
as follows. The invariant trace field $k$ of $M$ will always be
contained in the field $K$ generated by the simplex parameters $z_i$,
$i=1,\dots,n$. The $j$-th component $\vol_j([M]_{PSL})$ of $\Borel(M)$
is $$\Borel(M)_j=\sum_{i=1}^n D_2(\tau_j(z_i)),$$ where $\tau_j\colon
K\to\C$ is any complex embedding that extends $\sigma_j\colon k\to
\C$. Here $D_2$ is the ``Wigner dilogarithm function'' $$D_2(z) =
\operatorname{Im} \ln_2(z) +
\log |z|\arg(1-z),\quad z\in \C -\{0,1\},$$ where $\ln_2(z)$ is the
classical dilogarithm function. $D_2(z)$ is also the volume of the
ideal simplex with parameter $z$.

As described earlier, \snap\ specifies the invariant trace field $k$
as a subfield of $\C$ by giving the minimal polynomial of a
``canonical'' primitive element together with the position of the this
primitive element in a list of the roots of this polynomial. \snap\
numbers the roots with non-negative imaginary part using real roots
first in order of size, say $c_1<c_2<\dots<c_{r_1}$, and then non-real
roots in lexicographic order of size of real and imaginary parts,
$c_{r_1+1},\dots,c_{r_1+r_2}$. Finally, roots with negative imaginary
part have negative indices: $c_{-j}=\overline c_j$. The ``canonical
element'' is the first complex root in the list $c_{r_1+1},\overline
c_{r_1+1},c_{r_1+2},\overline c_{r_1+2},\dots$ that generates the
correct subfield of $\C$.

In printing $\Borel(M)$ \snap\ uses the complex embeddings given by
the complex roots $c_{r_1+1}, c_{r_1+2}, \dots$ above. The effect is
that, according as the canonical element is $c_{r_1+j}$ or
$c_{-(r_1+j)}$, the component $\Borel(M)_j$ of the Borel regulator is
$\vol(M)$ or $-\vol(M)$. In the latter case --- more generally,
whenever $k\ne\overline k$ --- the Borel regulator $\Borel(-M)$ is
simply $-\Borel(M)$. However, if $k=\overline k$ then \snap's printout
of $\Borel(M)$ and $\Borel(-M)$ refer to the same embedding of $k$
(both times given by the same canonical element), so the relation is
given by the action of conjugation on $\Bloch(k)$, which is a bit more
subtle.

It can be shown that $\pm \vol(M)$ is, in fact, the component with
largest absolute value in the Borel regulator (see
\cite{neumann-yang3}).

Some interesting examples with invariant trace field
$\Q(x)/(x^4+x^2-x+1)$ are discussed in \cite{neumann-yang3}. We list
all examples with this invariant trace field from the closed and
cusped censuses in Table 5.\iffalse The Bloch group has rank $2$ since the
field has two complex embeddings. The first complex embedding occurs
for $11$ different manifolds in the censuses, giving three different
Borel regulators, which are, however, proportional to each other. For
the second embedding there are $13$ manifolds, giving four different
Borel regulators, only two of which are proportional to each other. \fi

To compare the Bloch invariants of manifolds with different trace
fields we must compute in the Bloch group of a common field. We close
this section with interesting examples which illustrate this. 

\def\vol3{\mbox{\it Vol3}}
\def\weeks{\mbox{\it Weeks}}

\begin{example} The manifold of conjecturally smallest volume is the
so-called Weeks manifold $\weeks$ which is $\hbox{m003}(-3,1)$ in the
closed census. Its invariant trace field is: $$[x^3
- x^2 + 1, -2],$$ by which we mean the subfield of $\C$ generated by
the complex conjugate of the second root of the polynomial $x^3-x^2+1$
(the first root is the real root). This field has one complex
embedding, so the Borel regulator has just one component, which, by
the above discussion, will be minus the volume:
$$\Borel(\weeks)=[-0.9427073627769277209212996031]$$ 

The manifold of conjecturally third smallest volume is called
$\hbox{m007}(3,1)$ in the closed census. It is an arithmetic manifold
of exactly half the volume of the figure eight knot complement, i.e.,
its volume is the volume $1.0149416..$ of a regular ideal simplex. Let
us call this manifold $\vol3$ for short. Its invariant trace field is
$$[x^2-x+1,1]$$ and its Borel regulator is thus
$$[1.014941606409653625021202554].$$ However, we can ask \snap\ to
compute the Borel regulator in the field $k(\weeks)=[x^3 - x^2 + 1,
-2]$ of the Weeks manifold instead. \snap\ complains that this field
does not contain our invariant trace field, and then proceeds to
compute the join of the two fields and gives us the answer in that
field: $$\displaylines{ [x^6 - x^5 + x^4 - 2x^3 + x^2 + 1, -2]\cr
[1.014941606409653625021202554, -1.014941606409653625021202554,\cr
-1.014941606409653625021202554].}$$ From this we see that the joined
field $K$ is degree $6$, as expected, and that it has three complex
embeddings and they restrict on $k(\vol3)$ once to the given embedding
and twice to its conjugate.

Computing with the Weeks manifold in this same field we get a Borel
regulator: $$[0, -0.9427073627769277209212996031,
0.9427073627769277209212996031]$$ (which tells us that the first
complex embedding of our degree 6 field restricts to the real
embedding of $k(\weeks)$ and the next two complex embedding restricts
to the complex embedding of $k(\weeks)$ and its conjugate). 

It has been asked if the Bloch group can be generated by Bloch
invariants of $3$-manifolds (a positive answer would imply the
``Rigidity Conjecture'', see e.g., \cite{neumann-yang3} and
\cite{neumann-hilbert}). If so, one might guess that a ``random''
3-manifold with invariant trace field equal to the above degree 6
field $K$ is likely to have Borel regulator linearly independent of
the above two Borel regulators, since the Bloch group has rank $3$.
There turn out to be just two manifolds in the closed census with this
invariant trace field (as far as has been computed). They are
$\hbox{v2274}(-3,2)$ and $-\hbox{v2274}(3,2)$, and they both have the same 
Borel regulator, namely: $$\displaylines{[2.029883212819307250042405108,
-4.858005301150090412806303917,\cr 0.7982388755114759127214937007].}$$
It turns out that this is, at least numerically, equal to
$$3\Borel(\weeks)+2\Borel(\vol3).$$
\end{example}

Other interesting examples are given by surgeries on the census
manifold \hbox{v3066}, as discussed in \cite{neumann-yang3}.  This
manifold gives some of the most interesting examples of the ``twins''
phenomenon discussed in Example \ref{twinsex}. The four surgeries
$\hbox{v3066}(\pm p,q)$ and $\hbox{v3066}(\pm 2q, p/2)$ all have the
same volume for each $p,q$.
\begin{example}\label{v3066}\def\K{K_{18}} The manifolds
$M_1=\hbox{v3066}(6,1)$ and $M_2=\hbox{v3066}(-6,1)$ have invariant
trace fields
$$[x^9 - 2x^7 - 5x^6 + 12x^5 + 8x^4 + 15x^3 + 4x^2 + 2x - 1,-2]$$ and
$$[x^9 - 2x^7 - 5x^6 + 12x^5 + 8x^4 + 15x^3 + 4x^2 + 2x - 1,-5]$$
respectively. The join of these fields is $$
\begin{aligned}\K=[x^{18}& - 6x^{16} - 4x^{15} + 8x^{14} + 6x^{13} +
19x^{12} + 16x^{11}\\& + 32x^{10} - 84x^9 - 104x^8 + 52x^7 + 67x^6 -
8x^5 + 30x^4 - 28x^3 + 8x^2 - 2x + 1,-1],\end{aligned} $$ with $9$
complex embeddings, and the Borel regulators of the above two
manifolds, computed in this join, are respectively:
$$\begin{aligned}
\beta_1&=[-2a_1-a_2,-a_1,2a_1+a_2,a_1+a_2,0,-a_2,a_1+a_2,-a_1,a_2]\\
\beta_2&=[-2a_1-a_2,a_2,a_1,a_1+a_2,-a_1-a_2,a_1,0,-2a_1-a_2,a_2],
\end{aligned}
$$ where $$a_1=2.568970600936708884920674169,\quad
a_2=0.6083226776636170914331534552. $$ The automorphism group of the
field $\K$ is order $6$. Each of $\beta_1$ and $\beta_2$ is fixed by
an involution in this automorphism group, since they come from degree
9 subfields. Nevertheless, we can find three Galois conjugates of each
of $\beta_1$ and $\beta_2$, so we might hope to generate up to a rank
6 subgroup of $\Bloch(\K)$. But in fact, we only generate a rank 3
subgroup.
\iffalse
\footnote{This is in keeping with the Ramakrishnan conjecture,
Conjecture \ref{rama}, which would imply that the first coordinate of
the Borel regulator should have rank 1 kernel in this case. Indeed, it
is shown in
\cite{neumann-yang2} (see also the proof of Theorem \ref{sc} below)
that the conjecture would imply the kernel of the $i$-th coordinate
has rank equal to the number of complex embeddings of the real
subfield of $\sigma_i(\K)$ (the $i$-th complex embedding). This
subfield is degree 3 with 1 complex embedding in our case.}. \fi

The Galois conjugates of $\beta_1$ are $\beta_1$ and $$\begin{aligned}
\beta_1'&=[-a_1,a_2,-a_2,0,-a_1-a_2,2a_1+a_2,-a_1-a_2,-2a_1-a_2,-a_2]\\
\beta_1''&=[a_2,-2a_1-a_2,a_1,-a_1-a_2,a_1+a_2,a_1,0,a_2,-2a_1-a_2]
\end{aligned}
$$ and we find that
$$\beta_2=\frac13(2\beta_1+2\beta_1'-\beta_1'').$$ 

Various 3-manifolds can be found in the census with invariant trace
fields contained in $\K$. So far they all have Bloch invariant in the
above rank 3 subgroup of $\Bloch(\K)$. For example the field
$[x^3+2x-1,2]$ is the fixed field of $\operatorname{Aut}(\K)$. It
occurs as the invariant trace field of various manifolds, for example
$\hbox{v3066}(1,2)$, and they all have Borel regulator computed in
$\K$ proportional to
$\Borel(\hbox{v3066}(1,2)=\frac23(\beta_1+\beta_1'+\beta_1'')$. The
field $[x^3 - x^2 + x + 1,i]$ occurs as a subfield of $\K$ for each of
its three embeddings $i=1,2,-2$. The real embedding ($i=1$) is in fact
the real subfield of $\K$. The complex embedding and its conjugate
occur for many census manifolds and leads to Borel regulators in $\K$
that are integer multiples of $2\beta_1-\beta_1'-\beta_1''$ or its
Galois conjugate $2\beta_1''-\beta_1-\beta_1'$, depending on
orientation. The third Galois conjugate $2\beta_1'-\beta_1''-\beta_1$
must belong to the embedding $[x^3-x^2+x+1,1]$, i.e., to the real
subfield of $\K$. We will use this fact in the next section. 

In addition to three embeddings of the degree $9$ field already
mentioned, the only other subfields of $\K$ are $\Q(\sqrt{-11})$ and
two degree $6$ fields (the joins of $\Q(\sqrt{-11})$ with the degree
three subfields above; one of these degree $6$ fields is Galois over
$\Q$). None of these degree $2$ and $6$ fields have been found so far
in the census. One must, however, be careful about making premature
guesses from these data: arithmetic manifolds exist for any imaginary
quadratic field --- for $\Q(\sqrt{-11})$ they have just not been found
in the census. The Bloch invariant for these arithmetic manifolds will
lie outside the above rank three subgroup of $\Bloch(\K)$.
\end{example}

\section{Scissors Congruence}

\def\sc{\operatorname{\mathcal P}}
\def\dehnker{\operatorname{\mathcal D}}

The \emph{scissors congruence group} $\sc(\H^3)$ is the abelian group
generated by congruence classes of hyperbolic polyhedra of finite
volume modulo all relations of the form: $P=P_1+\dots+P_n$ if the
polyhedra $P_1,\dots,P_n$ can be glued along faces to create the
polyhedron $P$. Dupont and Sah showed that one obtains the same group
whether one allows ideal polyhedra or not (\cite{dupont-sah}; for an
exposition and references for the material of this section see
\cite{neumann-hilbert}).

The \emph{Dehn invariant} is the map
$$\delta\colon\sc(\H^3)\to\R\otimes\R/\pi$$ defined on generators of
$\sc(\H^3)$ as follows. If $P$ is a compact polyhedron then
$\delta(P)=\sum_El(E)\otimes\theta(E)$ where the sum is over the edges
$E$ of $P$ and $l(E)$ and $\theta(E)$ are length and dihedral angle.
For an ideal polyhedron one first truncates the ideal vertices by
horocycles and then uses the same definition, summing only over edges
that do not bound one of the horocycle faces of the truncated
polyhedron. The kernel of the Dehn invariant will be denoted
$$\dehnker(\H^3):=\ker(\delta\colon\sc(\H^3)\to\R\otimes\R/\pi).$$ 

If one subdivides an hyperbolic 3-manifold $M$ into polyhedra then the
sum of these polyhedra defines an element $\beta_0(M)$ in the scissors
congruence group $\sc(\H^3)$ and it is an easy exercise to see that in
fact $\beta_0(M)$ is in $\dehnker(\H^3)$. 

This group $\dehnker(\H^3)$ is closely related to the Bloch group.
Since $\Bloch(\C)$ is a $\Q$-vector space, it splits as the direct sum
$$\Bloch(\C)=\Bloch_+(\C)\oplus\Bloch_-(\C)$$ of its $+1$ and $-1$
eigenspaces under the action of conjugation. Dupont and Sah
\cite{dupont-sah} showed:
\begin{theorem} The Dehn invariant kernel
$\dehnker(\H^3)$ is naturally isomorphic to $\Bloch_-(\C)$. In fact
the natural map of the pre-Bloch group $\Prebloch(\C)$ to $\sc(\H^3)$,
defined by mapping a class $[z]$ to the ideal simplex with parameter
$z$, induces a surjection $\Bloch(\C)\to\dehnker(\H^3)$ with kernel
$\Bloch_+(\C)$. The Bloch invariant $\beta(M)$ is taken to the
scissors congruence class $\beta_0(M)$ by this map. \end{theorem}

In particular, this implies that the scissors congruence class
$\beta_0(M)$ is orientation-insensitive. In fact, it was first pointed
out by Gerling in a letter to Gauss that any polyhedron is scissors
congruent to its mirror image. The paper \cite{neumann-hilbert}
discusses to what extent one may think of the Bloch group as giving an
orientation-sensitive version of scissors congruence, and in
\cite{neumann-yang3} an explicit interpretation in terms of scissors
congruence allowing only cut-and-paste along ideal triangles is
described. However, the geometric interpretation of this for
$\beta(M)$ needs care --- for instance the manifold $\vol3$ discussed
earlier appears to have no subdivision into ideal tetrahedra at all. 

Note that if two manifolds have the same scissors congruence class,
say $\beta_0(M_1)=\beta_0(M_2)$, this means {\it a priori} only that
$M_1$ and $M_2$ are \emph{stably} scissors congruent; that is, there
is some polyhedron $Q$ such that $M_1+Q$ can be cut-and-pasted to form
$M_2+Q$. However, one can show that if $M_1$ and $M_2$ are either both
compact or both non-compact then adding $Q$ is unnecessary: $M_1$ can
be cut into polyhedra that can be reassembled to form $M_2$. 

\begin{theorem}\label{sc} Suppose $M_1$ and $M_2$ both have invariant
trace field contained in the field $K$. The following are equivalent: 

1.~~$M_1$ and $M_2$ are stably scissors congruent, that is
$\beta_0(M_1)=\beta_0(M_2)$.

2.~~$\Borel(M_1)+\Borel(-M_1)=\Borel(M_2)+\Borel(-M_2)$ (this must be
computed over a field containing $K$ and $\overline K$). 

3.~~$\Borel(M_1)-\Borel(M_2)$ is proportional to some $\Borel(x)$ with
$x\in\Bloch(K\cap\R)$.
\end{theorem}
\begin{proof} The equivalence of the first two conditions follows
because $\beta(-M)=-\overline\beta(M)$ and the map
$x\mapsto\frac12(x-\overline x)$ defines the projection
$\Bloch(\C)\to\Bloch_-(\C)$.

Denote $\Bloch(K)_\Q$ the image of $\Bloch(K)\otimes\Q$ in
$\Bloch(\C)\otimes\Q=\Bloch(\C)$ (recall $\Bloch(\C)$ is a $\Q$-vector
space). In \cite{neumann-yang2} it is shown that the
$\Bloch(K)_\Q\cap\Bloch_+(\C)=\Bloch(K\cap\R)_\Q$. This is thus the
kernel of the map $\Bloch(K)\to\sc(\H^3)$, proving equivalence of the
third condition.
\end{proof}

\begin{example}\def\K{K_{18}} Returning to the manifolds
$M_1=\hbox{v3066}(6,1)$ and $M_2=\hbox{v3066}(-6,1)$ of Example
\ref{v3066}, we find that they are scissors congruent. Indeed,
$\Borel(M_1)-\Borel(M_2)=
[0,-a_1-a_2,a_1+a_2,0,a_1+a_2,-a_1-a_2,a_1+a_2,-a_1-a_2,0]=
\frac13(2\beta_1'-\beta_1''-\beta_1)$, and we pointed out in Example
\ref{v3066} that this Borel regulator comes from the real subfield of
$\K$. \end{example}

The following conjecture has been made by many people. It is, as
discussed in \cite{neumann-hilbert}, also a consequence of Conjecture
\ref{rama} and hence of the Ramakrishnan conjecture.
\begin{conjecture} The map $\mbox{{\rm vol}}\colon\dehnker(\H^3)\to\R$ is
injective. \end{conjecture}

\snap\ provides many examples like the above which 
give evidence for this conjecture.

\section{Some tables}\label{tables}

The tables in this section list some arithmetic and numerical invariants
of hyperbolic 3-manifolds computed using Snap. Much more extensive tables of
results are available from http://www.ms.unimelb.edu.au/\~{}snap.

Under the heading ``Invariant trace field'' we list:
the canonical minimal polynomial $p$ defining the field,
the signature $[r_1,r_2]$, and the canonical root
number (as described in footnote \ref{rootnumbers}).

Under the heading ``Quaternion algebra'' we list the
finite ramification (giving generators for the corresponding
prime ideals), then real ramification of the
invariant quaternion algebra (giving the root number for the
corresponding real field embeddings). 
The last column Int/Ar indicates whether all
traces are integral and whether manifold is arithmetic
(1 = yes, 0 = no).
Manifolds are named using the notation of SnapPea;
$*$ is used to denote the mirror image of a manifold.


Table~\ref{lowvolclosed} lists invariants for the first 12 closed hyperbolic
3-manifolds in the Hodgson-Weeks census \cite{Ho-We}. These are conjectured 
to be the 12 hyperbolic 3-manifolds of smallest volume.


Table~\ref{algebras} includes examples of closed manifolds
chosen to illustrate various phenomena including

\noindent$\bullet$~~~~manifolds with the same invariant trace field
but different invariant quaterion algebras,

\noindent$\bullet$~~~~closed manifolds with the full matrix
algebra as invariant quaternion algebra (i.e. no ramification),

\noindent$\bullet$~~~~ arithmetic and non-arithmetic manifolds with the
same invariant quaterion algebra,

\noindent$\bullet$~~~~ manifolds with the same abstract invariant trace field, 
but different complex embeddings,

\noindent$\bullet$~~~~manifolds with the same invariant quaternion algebra,
but not commensurable (distingushed by integrality of traces).


For cusped manifolds, the invariant quaternion algebra is always the
full matrix algebra over the invariant trace field. 
For non-arithmetic cusped manifolds with one cusp, we list
another useful commensurability invariant: the density of a
maximal embedded horoball cusp (see \cite{nr2}).  
A similar invariant can be
defined for multicusped cusped non-arithmetic manifolds, provided
that there is a finite sheeted covering where all cusps are equivalent 
under the symmetry group. In this case, we compute the cusp density by taking
equal volume horoballs at all the cusps.

Table~\ref{cusped_ex} includes examples of cusped manifolds chosen
to ilustrate various phenomena including:

\noindent$\bullet$~~~~ arithmetic and non-arithmetic manifolds with the
same invariant quaterion algebra,

\noindent$\bullet$~~~~ non-arithmetic manifolds with the same 
invariant quaternion algebra but different cusp densities,

\noindent$\bullet$~~~~ manifolds with the same abstract invariant trace field, 
but different complex embeddings.

This table includes some familiar knots complements: 
$\hbox{m004}, \hbox{m015}, \hbox{m016}, \hbox{m032}$ are the 
complements of knots $4_1$, $5_2$, the $-2,3,7$-pretzel, and knot $6_1$
respectively. A table of arithmetic invariants computed using Snap 
for the complements of knots with up to 8 crossings is given in 
\cite{call-reid}.


Table~\ref{bloch_exs} lists Borel regulators and arithmetic 
invariants for all the closed and cusped census manifolds for which
the invariant trace field has been computed to be $x^4+x^2-x+1$.
Some of these examples are discussed in \cite{neumann-yang3}. Note that
the first two Borel regulators are proportional for the field with root 2,
while all three Borel regulators are proportional for the field with root 1.
The table also includes examples of the following phenomena:

\noindent$\bullet$~~~~manifolds with same Borel regulator 
but different invariant quaternion algebras,

\noindent$\bullet$~~~~closed and cusped manifolds with the same
Borel regulator,

\noindent$\bullet$~~~~ manifolds $\hbox{v2050}(4,1)$ and 
$\hbox{v3404}(1,3)$ with the
same arithmetic invariants (invariant trace field, 
invariant quaternion algebra, non-integral traces) but not commensurable
as their Borel regulators are not proportional.

\begin{sidewaystable}
\centering
\begin{tabular}{|>{\PreserveBackslash\raggedright}p{4.2cm}| %
>{\PreserveBackslash\raggedright}p{4.2cm}|
	>{\PreserveBackslash\raggedright}p{6cm}|
	>{\PreserveBackslash\raggedright}p{2.4cm}|l|}

\hline

\hbox{Manifold} Volume & \hbox{Eta invariant} Chern-Simons (mod 1)
& Invariant trace field & \hbox{Quaternion} algebra & Int/Ar\ \\

\hline 

m003(-3,1)
0.94270736277692772092 &
0.04002871111915143667 
0.06004306667872715501 &
\hbox{$x^3 - x^2 + 1$} [1, 1] (-2) &
\hbox{$(5, x-2)$} [1]&
$1/1$ \\

\hline 

m003(-2,3)
0.98136882889223208809 &
0.71802545350918014836 
0.07703818026377022254 &
\hbox{$x^4 - x - 1$} [2, 1] (3) &
[1,2] &
$1/1$ \\

% 1.0149416064096536        Z/3 + Z/6  0.0000000000000000  0.831442945529  m007( 3, 1)

\hline 

m007(3,1)
1.01494160640965362502 &
0.00000000000000000000
-0.50000000000000000000 &
\hbox{$x^2 - x + 1$} [0, 1] (1) &
\hbox{$(2) (3, x+1)$} [~]&
$1/1$ \\

% 1.2637092386580437        Z/5 + Z/5 -0.1141406559693699  0.575078580850  m003(-4, 3)


\hline 

m003(-4,3)
1.26370923865804365588 &
0.92390622935375341671 
0.38585934403063012507 &
\hbox{$x^4 - x^3 + x^2 + x - 1$} [2, 1] (-3) &
[1,2] &
$1/1$ \\



% 1.2844853004683544              Z/6  0.0679316734798631  0.480311803385  m004( 6, 1)

\hline 

m004(6,1)
1.28448530046835444246 &
1.04528778231990871951 
0.06793167347986307927 &
\hbox{$x^3 + 2x - 1$} [1, 1] (2) &
\hbox{$(2, x^2 + x + 1)$} [1]&
$1/1$ \\


% 1.3985088841508066          trivial -0.2466072526507687  0.366130702346  m004( 1, 2)

\hline 

m004(1,2)
1.39850888415080664050 &
-0.83107150176717910541 
-0.24660725265076865812 &
\hbox{$x^7 - 2x^6 - 3x^5 + 3x^4 + 5x^3 - x^2 - 3x + 1$} [5, 1] (6) &
[2, 3, 4, 5] &
$1/0$ \\



\hline 

% 1.4140610441653916              Z/6  0.0766020666485759  0.794134662992  m009( 4, 1)
m009(4,1)
1.41406104416539158138 &
0.38440137776571728943 
0.07660206664857593414 &
\hbox{$x^3 - x^2 + 1$} [1, 1] (2) &
\hbox{$(5, x-2)$} [1]&
$1/1$ \\


\hline 

% 1.4140610441653916             Z/10  0.1182687333152426  0.364894686526  m003(-3, 4)
m003(-3,4)
1.41406104416539158138 &
0.41217915554349506721 
0.11826873331524260081 &
\hbox{$x^3 - x^2 + 1$} [1, 1] (2) &
\hbox{$(19, x-3)$} [1]&
$1/1$ \\



\hline 

% 1.4236119002928252             Z/35  0.1125651520461811  0.352371598411  m003(-4, 1)
m003(-4,1)
1.42361190029282524980 &
-0.25828989863587927861 
-0.38743484795381891791 &
\hbox{$x^5 - x^3 - x^2 + x + 1$} [1, 2] (2) &
\hbox{$(13, x+5)$} [1]&
$1/0$ \\



\hline 

% 1.4406990067273649              Z/3 -0.2400607094618030  0.361521576120  m004( 3, 2)
m004(3,2)
1.44069900672736487528 &
-0.49337380630786866586 
0.25993929053819700120 &
\hbox{$x^6 - x^5 - 2x^4 - 3x^3 + 3x^2 + 3x - 2$} [4, 1] (5) &
\hbox{$(2,x)$} [1, 3, 4] &
$1/0$ \\



\hline 

% 1.4637766449272388              Z/7  0.0606168663471378  0.356645033638  m004( 7, 1)
m004(7,1)
1.46377664492723877337 &
1.37374457756475854543 
0.06061686634713781814 &
\hbox{$x^6 - x^5 + x^4 + 2x^3 - 4x^2 + 3x - 1$} [2, 2] (3) &
[1, 2] &
$1/0$ \\


\hline 

% 1.5294773294300263              Z/5 -0.2346223683614192  0.335975531076  m004( 5, 2)
m004(5,2)
1.52947732943002626282 &
-0.15641491224094610942 
-0.23462236836141916413 &
\hbox{$x^7 - x^6 - 2x^5 + 5x^4 - 6x^2 + x + 1$} [5, 1] (6) &
[2, 3, 4, 5] &
$1/0$ \\

\hline 

\end{tabular}
\caption{Arithmetic invariants for the first 12 manifolds
from the Hodgson-Weeks closed census}
\label{lowvolclosed}
\end{sidewaystable}

\begin{table}[p] 
\centering
\begin{tabular}{|>{\PreserveBackslash\raggedright}p{3.5cm}
	|>{\PreserveBackslash\raggedright}p{3.5cm}|l %
	|>{\PreserveBackslash\raggedright}p{5cm}|}

\hline

Invariant trace field & Quaternion algebra &  Int/Ar\ & Manifolds \\

\hline

\hbox{$x^2 + 1$} [0, 1] (1) &
\hbox{$(2, x + 1) (3)$} [~]&
$1/1$ &
{m304(5,1) m336(-1,3) s942(-2,1) s960(-1,2)} \\

\cline{2-4}

& % $x^2 + 1$ [0, 1] (1) &
\hbox{$(2, x + 1) (5, x - 2)$} [~]&
$1/1$ &
{m293(4,1) s297(-1,3) s572(1,2) s645(-1,2) s682(-3,1) s775(-1,2) s778(-3,1) v3213(-1,3) v3216(4,1)} \\

\cline{2-4}

& % $x^2 + 1$ [0, 1] (1) &
\hbox{$(2, x + 1) (5, x + 2)$} [~]&
$1/1$ &
{m006(1,3) m009(-5,1) m009(5,1) m010(-2,3) m294(4,1) m312(-1,3) s296(5,1) s350(-4,1) s495(1,2) s595(3,1) s775(-3,1) s779(2,1) v3217(-1,3) v3412(5,1)} \\

\cline{2-4}

& % $x^2 + 1$ [0, 1] (1) &
[~] &
$0/0$ &
{m239(-2,3) s254(-3,2)} \\

\hline

\hbox{$x^2 - x + 2$} [0, 1] (1) &
\hbox{$(2, x) (7, x + 3)$} [~]&
$1/1$ &
{m140(-4,1) v3110(3,1) v3147(-3,1)} \\

\cline{3-4}

& % $x^2 - x + 2$ [0, 1] (1) &
& % $[~] (2, x) (7, x + 3)$ &
$0/0$ &
{v3377(-3,1) v3378(-3,1) v3390(3,1)} \\

\hline

\hbox{$x^3 + 2x - 1$} [1, 1] (2) &
\hbox{$(2, x^2 + x + 1)$} [1]&
$1/1$ &
{m004(6,1) m160(1,2) m306(-5,1) m307(-1,3) s554(3,1) s594(-3,2)* v2642(5,1) v2643(-2,3)} \\

\cline{2-4}

& % $x^3 + 2x - 1$ [1, 1] (2) &
\hbox{$(2, x + 3) (2, x^2 + x + 1)$} [~]&
$1/0$ &
{m136(1,2) v2920(-1,2)* v3066(1,2) v3528(3,1)} \\

\hline

\hbox{$x^3 + x - 1$} [1, 1] (2) &
[~] &
$1/0$ &
{s772(-5,2) s772(3,2)* s775(-5,2) s775(3,2)* s778(-5,2) s778(3,2)* s779(-5,2) s779(3,2)* s787(-5,2) s787(3,2)*} \\

\hline

\hbox{$x^3 - x - 2$} [1, 1] (2) &
\hbox{$(2, x + 1)$} [1]&
$0/0$ &
{m293(-2,3)* m390(3,1)*} \\

\cline{3-4}

& % $x^3 - x - 2$ [1, 1] (2) &
& % $[1] (2, x + 1)$ &
$1/1$ &
{m307(-5,1)* m369(-1,3) m371(1,3)* s298(5,1) s594(1,2)* s594(2,1)} \\

\cline{2-4}

& % $x^3 - x - 2$ [1, 1] (2) &
\hbox{$(2, x + 1) (2, x)$} [~]&
$1/0$ &
{s235(-4,3) s595(1,2)} \\

\hline

\hbox{$x^4 - 2x^3 - x^2 + 2x + 2$} [0, 2] (2) &
\hbox{$(13, x + 2) (13, x - 3)$} [~]&
$0/0$ &
{v3207(5,1) v3209(4,3) v3210(5,1) v3208(4,3)} \\

\hline

\hbox{$x^4 + x^2 - x + 1$} [0, 2] (2) &
[~] &
$1/0$ &
{s594(-3,4)* s594(-4,3) } \\ 

\cline{3-4}

& % $x^4 + x^2 - x + 1$ [0, 2] (1) &
& % $[~]$ &
$0/0$ &
{v2050(4,1)* v3404(1,3)} \\

\hline

\hbox{$x^4 + x^2 - x + 1$} [0, 2] (1) &
[~] &
$1/0$ &
{m010(-1,3) m368(4,1) m369(3,1)* m370(-4,1)* s313(-2,3)* s554(1,3)} \\

\hline

\hbox{$x^5 - x - 1$} [1, 2] (2) &
\hbox{$(2, x^3 + x^2 + 1)$} [1]&
$1/0$ &
{v3221(1,2) v3228(-1,2)*} \\

\hline

\hbox{$x^5 - x - 1$} [1, 2] (3) &
\hbox{$(2, x^3 + x^2 + 1)$} [1]&
$1/0$ &
{v3100(1,3)} \\

\hline

\end{tabular}
%\end{center}
\caption{Arithmetic invariants of some selected manifolds from the Hodgson-Weeks
closed census.}
\label{algebras}
\end{table}

\begin{table}[p] 
%\begin{center}
\centering  
\begin{tabular}{|>{\PreserveBackslash\raggedright}p{3.5cm} |l %
	|l %>{\PreserveBackslash\raggedright}p{3.5cm}
	|>{\PreserveBackslash\raggedright}p{5cm}|}

\hline

Invariant trace field &  Int/Ar\ & Cusp Density & Manifolds \\

\hline

\hbox{$x^2 + 1$} [0, 1] (1) &
$1/1$ &
~~&
{m001 m124 m125 m126 m127 m128 m129 m130 m131 m132 m133 m134 m135 m136 m139 m140 
	s859 v1858 v1859} \\

\cline{2-4}

& % $x^2 + 1$ [0, 1] (1) &
$0/0$ &
 0.614106035 &
 {m137 m138} \\

\hline

%Field: x^2 - x + 1 (1)   (37 manifolds)

\hbox{$x^2 - x + 1$} [0, 1] (1) &
$1/1$ &
~~&
{m000 m002 m003 m004 m025 m202 m203 m204 m205 m206 m207 m208 m405 m406  
m407 m408 m409 m410 m411  m412 m413 m414  s118 s119 s594 s595 s596
 s955 s956 s957 s958 s959 s960  v2873 v2874 v3551} \\

\cline{2-4}

& % $x^2 - x + 1$} [0, 1] (1) &
$0/0$ &
 0.568850725 &
{v2875} \\

\hline

%Field: x^2 - x + 2 (1)    (22 manifolds)

\hbox{$x^2 - x + 2$} [0, 1] (1) &
$1/1$ &
~~&
{m009 m010 s772 s773  s774 s775 s776 s777 s778 s779 s780 s781 s782 s784
 s786 s787} \\

\cline{2-4}

& % $x^2 - x + 2$} [0, 1] (1) &
$0/0$ &
 N/A~(inequivalent cusps)&   %symmetries don't interchange the cusps
{s785} \\

\cline{2-4}

& % $x^2 - x + 2$} [0, 1] (1) &
$0/0$ &
 0.558071819&
{s783} \\

\cline{2-4}

& % $x^2 - x + 2$} [0, 1] (1) &
$0/0$ &
 0.620079799 &
{s788 s789 v1539 v1540} \\

\hline

%Field: x^3 - x^2 + 1 (2)    (17 manifolds)


\hbox{$x^3 - x^2 + 1$} [1, 1] (2) &
$1/0$ &
0.511270966&
{s898 v2202* v2203} \\ %s898 has two equivalent cusps

\cline{2-4}

& %\hbox{$x^3 - x^2 + 1$} [1, 1] (2) &
$1/0$ &
0.524808681&
{v3428*} \\

\cline{2-4}

& %\hbox{$x^3 - x^2 + 1$} [1, 1] (2) &
$1/0$ &
0.539001522 &  %v3429 has two equivalent cusps 0.5390015228575017
{ v3429*} \\

\cline{2-4}


& %\hbox{$x^3 - x^2 + 1$} [1, 1] (2) &
$1/0$ &
0.545958189&
{v0769} \\

\cline{2-4}

& %\hbox{$x^3 - x^2 + 1$} [1, 1] (2) &
$1/0$ &
0.575271908&
{s420*} \\

\cline{2-4}

& %\hbox{$x^3 - x^2 + 1$} [1, 1] (2) &
$1/0$ &
0.604035858&
{v3426 v3427} \\ %v3426 has two equivalent cusps

\cline{2-4}

& %\hbox{$x^3 - x^2 + 1$} [1, 1] (2) &
$1/0$ &
0.612276793&
{v2204* v2205*} \\

\cline{2-4}

& %\hbox{$x^3 - x^2 + 1$} [1, 1] (2) &
$1/0$ &
0.697799972&
{m015* m017* s899 s900} \\

\cline{2-4}

& %\hbox{$x^3 - x^2 + 1$} [1, 1] (2) &
$1/0$ &
0.711685428&
{m016* s897} \\

\hline


%Field: x^3 - x^2 + x + 1 (2)   (30 manifolds)

\hbox{$x^3 - x^2 + x + 1$} [1, 1] (2) &
$1/0$ &
N/A~(inequivalent cusps)&
{v3220 v3223* } \\

\cline{2-4}
& %\hbox{$x^3 - x^2 + x + 1$} [1, 1] (2) &
$0/0$ &
N/A~(inequivalent cusps)&
{v3224*} \\

\cline{2-4}

& %\hbox{$x^3 - x^2 + x + 1$} [1, 1] (2) &
$1/0$ &
0.616691512&
{m035 m037 m039* m040* v3218 v3222*  v3225* v3227*} \\
%v3218, v3222, v3225 have 2 equiv cusps
%v3227 has a 2-fold cover with symm gp = Z2 x D4 and 4 equiv cusps

\cline{2-4}

& %\hbox{$x^3 - x^2 + x + 1$} [1, 1] (2) &
$1/0$ &
0.623017665&
{m376*} \\

\cline{2-4}

& %\hbox{$x^3 - x^2 + x + 1$} [1, 1] (2) &
$1/0$ &
0.645539037&
{m036* m038 v3214 v3215* v3216 v3217*} \\

\cline{2-4}

& %\hbox{$x^3 - x^2 + x + 1$} [1, 1] (2) &
$0/0$ &
0.646337229 &
{v3226} \\

\cline{2-4}

& %\hbox{$x^3 - x^2 + x + 1$} [1, 1] (2) &
$0/0$ &
0.652161114 &
{s448} \\

\cline{2-4}

& %\hbox{$x^3 - x^2 + x + 1$} [1, 1] (2) &
$1/0$ &
0.675735988 &
{v3207 v3208 v3209 v3210} \\

\cline{2-4}

& %\hbox{$x^3 - x^2 + x + 1$} [1, 1] (2) &
$1/0$ &
0.717278605 &
{v3219 v3221 v3228*} \\

\cline{2-4}

& %\hbox{$x^3 - x^2 + x + 1$} [1, 1] (2) &
$1/0$ &
0.726163222 & %v3211  has 2 equiv cusps
{v3211 v3212 v3213*} \\

\hline


%%(two different embeddings of same field)

\hbox{$x^4 + x^2 - x + 1$} [0, 2] (1) &
$1/0$ &
 0.614493011&
{m161*} \\

\cline{2-4}

& %\hbox{$x^4 + x^2 - x + 1$} [0, 2] (1) &
$1/0$ &
 0.631076941&
{s919*} \\

\cline{2-4}

& %\hbox{$x^4 + x^2 - x + 1$} [0, 2] (1) &
$1/0$ &
 0.662737952&
{m159* m160} \\

\hline

\hbox{$x^4 + x^2 - x + 1$} [0, 2] (2) &
$1/0$ &
 0.595110801&
{s235} \\

\cline{2-4}

& %\hbox{$x^4 + x^2 - x + 1$} [0, 2] (2) &
$1/0$ &
 0.630681177&
{m032* m033*} \\

\cline{2-4}

& %\hbox{$x^4 + x^2 - x + 1$} [0, 2] (2) &
$1/0$ &
 0.686680170&
{s435* s436*} \\

\hline

\end{tabular}
%\end{center}
\caption{Arithmetic invariants of some selected manifolds from the 
Hildebrand-Weeks cusped census.}
\label{cusped_ex}
\end{table}






%\begin{figure}[p] 
\begin{sidewaystable}
\centering
\begin{tabular}{|>{\PreserveBackslash\raggedright}p{3.2cm}
|>{\PreserveBackslash\raggedright}p{4.2cm}
|>{\PreserveBackslash\raggedright}p{3.4cm}|l
|>{\PreserveBackslash\raggedright}p{2.0cm}
|>{\PreserveBackslash\raggedright}p{4.2cm}|}

\hline

Invariant trace field & Borel Regulator & Quaternion algebra &  Int/Ar\
& Manifolds &Chern-Simons (mod $\frac{1}{2}$) \\

\hline

\hbox{$x^4 + x^2 - x + 1 $} (2) &
-1.41510489726556334068 
3.16396322888314398399 &
$(2, x+3)$ ~$(13, x+6)$  [~]&
$1/0$ &
{m140(5,2)* }& 
0.17735631658981817209 \\

\cline{3-6}
 &
 &
$(2, x+3)$ ~$(2, x^3+x^2+1)$ [~]&
$1/0$ &
{m136(5,2)* }&
0.21902298325648483876 \\

\cline{3-6}
 &
 &
$(2, x+3)$ ~$(233, x+72)$ [~]&
$1/0$ &
{m140(-5,2) } &
-0.23931035007684849456 \\

\cline{3-6}
 &
 &
[~]&
$1/0$ &
{m032*} &
-0.15597701674351516123 \\

\cline{5-6}
 &
 &
&%[~]&
&%$1/0$ &
{m033*} &
0.09402298325648483876 \\

\cline{2-6}
 &
-2.12265734589834501103 
4.74594484332471597598 &
[~]&
$1/0$ &
{s435*} &
0.05770114155139392481 \\
\cline{5-6}
 &
&%-2.122657345898345011034262881 4.745944843324715975986522074 &
&%[~]&
&%$1/0$ &
{s436*} &
-0.19229885844860607518\\



\cline{2-6}
& %\hbox{$x^4 + x^2 - x + 1 $} (2) &
-0.21181355280835614147 
4.39667280193249561612 &
$(2, x+3)$ ~$(19, x-5)$ [~]&
$1/0$ &
{s855(3,2)} &
-0.24238579181095171467 \\

\cline{3-6}
& %\hbox{$x^4 + x^2 - x + 1 $} (2) &
&%-0.2118135528083561414788017694 4.396672801932495616121158767 &
 [~]&
$1/0$ &
{s594(-3,4)* } &
0.00761420818904828532 \\
\cline{5-6}
& %\hbox{$x^4 + x^2 - x + 1 $} (2) &
&%-0.2118135528083561414788017694 4.396672801932495616121158767 &
&% [~]&
&%$1/0$ &
{s594(-4,3) } &
-0.24238579181095171467 \\

\cline{5-6}
& %\hbox{$x^4 + x^2 - x + 1 $} (2) &
&%-0.2118135528083561414788017694 4.396672801932495616121158767 &
&% [~]&
&%$1/0$ &
{s235 } &
0.13261420818904828532 \\

\cline{4-6}
& %\hbox{$x^4 + x^2 - x + 1 $} (2) &
&%-0.2118135528083561414788017694 4.396672801932495616121158767 &
&% [~]&
$0/0$ &
{v2050(4,1)* } &
-0.20071912514428504800  \\

\cline{2-6}
& %\hbox{$x^4 + x^2 - x + 1 $} (2) &
0.99147779164885105773
5.62938237498184724825  &
[~]&
$0/0$ &
{v3404(1,3)} &
-0.16212790021172160144  \\

\hline
\hbox{$x^4 + x^2 - x + 1 $} (1) &
1.91084379308998886955  
-0.34927204139222035986 &
 [~]&
$1/0$ &
{m010(-1,3)}& 
-0.09574639997098769384\\

\cline{2-6}
&%\hbox{$x^4 + x^2 - x + 1 $} (1) &
3.82168758617997773911  
-0.69854408278444071973 &
$(2,x+3) (29, x-14)$ [~] &
$1/0$ &
{m294(4,3)*}& 
-0.02482613327530872102\\

\cline{5-6}

&%\hbox{$x^4 + x^2 - x + 1 $} (1) &
&%3.821687586179977739110922224  -0.6985440827844407197307266120 &
&%(2,3+x) (29, -14+x) [~] &
&%$1/0$ &
{m293(-4,1)}& 
0.22517386672469127897 \\

\cline{3-6}
&%\hbox{$x^4 + x^2 - x + 1 $} (1) &
&%3.821687586179977739110922224  -0.6985440827844407197307266120 &
 [~] &
$1/0$ &
{m369(3,1)* m370(-4,1)* m368(4,1) s554(1,3)}& 
0.14184053339135794563\\

\cline{5-6}
&%\hbox{$x^4 + x^2 - x + 1 $} (1) &
&%3.821687586179977739110922224  -0.6985440827844407197307266120 &
&% [~] &
&%$1/0$ &
{m160 m161*}& 
-0.19149279994197538769 \\

\cline{5-6}
&%\hbox{$x^4 + x^2 - x + 1 $} (1) &
&%3.821687586179977739110922224  -0.6985440827844407197307266120 &
&% [~] &
&%$1/0$ &
{m159*}& 
0.05850720005802461230 \\

\cline{2-6}
&%\hbox{$x^4 + x^2 - x + 1 $} (1) &
5.73253137926996660866 
-1.04781612417666107959 &
 [~] &
$1/0$ &
{s919*}& 
-0.16223919991296308154 \\

\hline

\end{tabular}
\caption{Bloch invariants of some closed and cusped manifolds with invariant trace field
$x^4 + x^2 - x + 1 $.}
\label{bloch_exs}
\end{sidewaystable}

\vfill
\eject

\begin{thebibliography}{99}

\bibitem{atiyah-patodi-singer2} 
M.~Atiyah, V.~Patodi and I.~Singer,
\newblock  Spectral asymmetry and Riemannian geometry 2,
\newblock Math. Proc. Cambridge Philos. Soc. {\bf78} (1975), 402--433. 

\bibitem{BPZ}
S.~Betley, J.H.~Przytycki and T.~Zukowski,
\newblock Hyperbolic structures on Dehn fillings of some 
punctured-torus bundles over $S^1$,
\newblock {Kobe J. Math.} {\bf 3} (1986), 117--147.

\bibitem{call-reid}
P.J.~Callahan and A.W.~Reid, 
\newblock Hyperbolic structures on knot complements,
\newblock Chaos, Solitons and Fractals, {\bf9} (1998), 705--738.

\bibitem{cheeger-simons} 
J.~Cheeger and J.~Simons, 
\newblock Differential characters and geometric invariants,
\newblock  Springer Lect. Notes in Math. {\bf1167} (1985), 50--80.

\bibitem{chern-simons}
S.~Chern, J.~Simons, 
\newblock {Some cohomology classes in principal fiber
bundles and their application to Riemannian geometry},
\newblock  Proc. Nat. Acad. Sci. U.S.A. {\bf68} (1971), 791--794.

\bibitem{pari}
H.~Cohen C.~Batut, D.~Bernardi and M.~Olivier,
\newblock {PARI-GP version 2},
\newblock {\tt ftp: megrez.math.u-bordeaux.fr}.


\bibitem{cohen}
H.~Cohen,
\newblock {\em A course in computational algebraic number theory},
\newblock Graduate Texts in Mathematics {\bf 138}, Springer-Verlag, 1993.

\bibitem{dupont-kamber} 
J.L.~Dupont and F.L.~Kamber,
\newblock Cheeger-Chern-Simons classes of transversally symmetric foliations:
dependance relations and eta-invariants,
\newblock  Math. Ann. {\bf295} (1993), 449--468.

\bibitem{dupont-sah}
J.L.~Dupont, H.~Sah,  Scissors congruences II, J.\ Pure and App.\
Algebra {\bf25} (1982), 159--195.

\bibitem{ep}
D.~B.~A.~Epstein and R.~C.~Penner,
\newblock Euclidean decompositions of hyperbolic manifolds,
\newblock { J. Diff. Geom.} {\bf27} (1988), 67--80.

\bibitem{gmmr}
 F.~W.~Gehring, C.~Maclachlan, G.~J.~Martin and A.~W.~Reid,
\newblock Arithmeticity, discreteness and volume,
\newblock Trans. Amer. Math. Soc. {\bf349} (1997), 3611--3643.

\bibitem{HiW} M.~Hildebrand and J.~Weeks,  A computer generated census of
cusped hyperbolic 3-manifolds, in: Computers and Mathematics, eds.
E.~Kaltofen and S.~Watt, Springer-Verlag (1989) 53-59.

\bibitem{hlm}
 H.M.~Hilden, M.T.~Lozano and J.M.~Montesinos-Amilibia,
\newblock {A characterization of arithmetic subgroups of $\mbox{SL}(2,{\mathbb
  R})$ and $\mbox{SL}(2,{\mathbb C})$},
\newblock { Math. Nachr.} {\bf159} (1992) 245--270.

\bibitem{Ho-We}
C.D.~Hodgson and J.~Weeks, A census of closed hyperbolic 3-manifolds,
in preparation. 

\bibitem{kirby}
R.~Kirby,
\newblock {Problems in low-dimensional topology},
\newblock {AMS/IP Studies in Advanced Mathematics},
\newblock {Volume 2, part 2,}
\newblock Amer. Math. Soc., 1997.

\bibitem{lang}
S.~Lang,
\newblock {\em Algebraic number theory},
\newblock Graduate Texts in Mathematics {\bf 110}, Springer-Verlag, 1986.

\bibitem{macbeath}
A.M.~Macbeath,
\newblock Commensurability of co-compact three-dimensional hyperbolic groups,
\newblock {Duke Math. J.} {\bf 50} (1983) 1245--1253.

\bibitem{magnus}
W.~Magnus,
\newblock {Rings of Fricke characters and automorphism groups of free groups},
\newblock { Math. Zeitschrift} {\bf 170} (1980), 91--103.

\bibitem{meyerhoff}
R.~Meyerhoff, 
\newblock  Hyperbolic 3-manifolds with equal volumes but
different Chern-Simons invariants,
\newblock in {\it Low-dimensional topology and
Kleinian groups}, edited by D.~B.~A.~Epstein, London Math.\ Soc.\
lecture notes series, {\bf112} (1986) 209--215.

\bibitem{meyerhoff-neumann} 
R.~Meyerhoff and W.~Neumann, 
\newblock An asymptotic formula for the $\eta$-invariant 
of hyperbolic 3-manifolds,
\newblock  Comment. Math. Helvetici {\bf67} (1992), 28--46.

\bibitem{meyerhoff-ouyang}
R.~Meyerhoff and M.~Ouyang, 
\newblock The $\eta$-invariants of cusped hyperbolic $3$-manifolds, 
\newblock Canad. Math. Bull. {\bf 40} (1997), 204--213.

\bibitem{milnor-stasheff} 
J.~W.~Milnor and J.~D.~Stasheff,
\newblock  {\em Characteristic classes}, 
\newblock Annals of Math. Studies {\bf76}, Princeton
University Press, 1974.

\bibitem{mostow}
G.~D.~Mostow,
\newblock {\em Strong rigidity of locally symmetric spaces}, 
\newblock Annals of Math. Studies {\bf78}, Princeton
University Press, 1973.

\bibitem{neumann1}
W.~D.~Neumann,
\newblock Homotopy invariance of Atiyah invariants,
\newblock  Proc. Symposia
in Pure Math., Volume 32, part 2, Amer. Math. Soc., 1978, 181--188.  


\bibitem{neumann2}
W.~D.~Neumann,  
\newblock Combinatorics of triangulations and the Chern
Simons invariant for hyperbolic 3-manifolds,
\newblock  in {\it Topology 90,
Proceedings of the Research Semester in Low Dimensional Topology at
Ohio State}, Walter de Gruyter Verlag, Berlin - New York 1992,
243--272.

\bibitem{neumann-hilbert} 
W.~D.~Neumann,
\newblock Hilbert's 3rd problem and 3-manifolds,
\newblock Geometry and Topology Monographs Volume 1: The Epstein Birthday
Schrift, Igor Rivin, Colin Rourke and Caroline Series (editors)
(International Press, 1998 and electronically at
www.maths.warwick.ac.uk/gt), 383--411.

\bibitem{neumann-in-progress}
W.~D.~Neumann,
\newblock {The extended Bloch group and the Chern-Simons invariant},
\newblock in preparation.

\bibitem{nr1}
W.~D.~Neumann and A.~W.~Reid,
\newblock {Arithmetic of hyperbolic manifolds},
\newblock In {\em Topology 90, Proceedings of the Research Semester in Low
  Dimensional Topology at Ohio State},  Walter de Gruyter Verlag,
  Berlin - New York, 1992, 273--310.

\bibitem{nr2}
W.~D.~Neumann and A.~W.~Reid,
\newblock {Notes on Adams' small volume orbifolds}
\newblock In {\em Topology 90, Proceedings of the Research Semester in Low
  Dimensional Topology at Ohio State},  Walter de Gruyter Verlag,
  Berlin - New York, 1992, 311--314.

\bibitem{neumann-yang1} W.~D.~Neumann and J.~Yang,
\newblock  Problems for
$K$-theory and Chern-Simons invariants of hyperbolic 3-manifolds,
\newblock L'Enseignement Math\-\'em\-at\-ique {\bf41} (1995), 281--296.

\bibitem{neumann-yang2}
W.~D.~Neumann and J.~Yang,
\newblock  Invariants from triangulation for
hyperbolic 3-manifolds,
\newblock  Electronic Research Announcements of the
Amer. Math. Soc. {\bf1} (2) (1995), 72--79.

\bibitem{neumann-yang3}
W.~D.~Neumann and J.~Yang,
\newblock  Bloch invariants of hyperbolic 3-manifolds,
\newblock Duke Math. J. (to appear).

\bibitem{nz}
W.~D.~Neumann, D.~Zagier,  Volumes of hyperbolic 3-manifolds,
Topology {\bf24} (1985), 307--332.

\bibitem{ouyang}
M.~Ouyang, 
\newblock A simplicial formula for the $\eta$-invariant 
of hyperbolic $3$-manifolds, 
\newblock Topology {\bf36} (1997), 411--421.

\bibitem{pohst}
Michael~E.~Pohst,
\newblock {\em Computational algebraic number theory}, { DMV
  Seminar} {\bf 21}, 
\newblock Birkh{\"{a}}user, 1993.

\bibitem{reidphd}
Alan~W.~Reid,
\newblock{Arithmetic Kleinian groups and their Fuchsian subgroups},
\newblock{ Ph.D. thesis}, Aberdeen, 1987.

\bibitem{reid}
Alan~W.~Reid,
\newblock A note on trace-fields of {Kleinian} groups,
\newblock { Bull. London Math. Soc.} {\bf 22} (1990), 349--352.

\bibitem{tak}
K.~Takeuchi,
\newblock {A characterization of arithmetic Fuchsian groups},
\newblock {J. Math. Soc. Japan} {\bf 27} (1975), 600--612.

\bibitem{thurston}
W.P.~Thurston,
\newblock {\em The geometry and topology of 3-manifolds},
\newblock Princeton University, 1977,
\newblock Mimeographed lecture notes.

\bibitem{vigneras}
M-F.~Vign\'{e}ras,
\newblock {\em Arithm\'{e}tique des alg\`{e}bres de quaternions},
\newblock Lecture Notes in Math. Springer-Verlag {\bf 800}, 1980.

\bibitem{snappea}
Jeffrey~R.~Weeks,
\newblock {SnapPea}, the computer program,
\newblock {\tt http://www.geom.umn.edu/software/\-down\-load/snappea.html}.

\bibitem{yoshida} 
T.~Yoshida,
\newblock {The $\eta$-invariant of hyperbolic 3-manifolds},
\newblock Invent. Math. {\bf81} (1985), 473--514.

\end{thebibliography}

\end{document}



